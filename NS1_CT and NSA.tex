\documentclass[reqno]{amsart}
%\usepackage[english]{babel}
\usepackage{amssymb,amsmath,hyperref}
\usepackage{amsrefs}


%%%%%%%%%%%%%%%%%%%%%%%%
% Dag Normann LaTeX definitions


%\newcommand{\R}{{\Bbb R}}
%\newcommand{\N}{{\Bbb N}}
%\newcommand{\B}{{\Bbb B}}
%\newcommand{\Q}{{\Bbb Q}}
%\newcommand{\Z}{{\Bbb Z}}
\newcommand{\lb}{[\![}
\newcommand{\rb}{]\!]}
\newcommand{\qcb}{\mathsf{QCB}}
\newcommand{\QCB}{\mathsf{QCB}}

%%%%%%%%%%%%%%%%%%%%%%%%%



\newtheorem{thm}{Theorem}
\newtheorem{lem}[thm]{Lemma}
\newtheorem{cor}[thm]{Corollary}
\newtheorem{defi}[thm]{Definition}
\newtheorem{rem}[thm]{Remark}
\newtheorem{nota}[thm]{Notation}
\newtheorem{exa}[thm]{Example}
\newtheorem{rul}[thm]{Rule}
\newtheorem{ax}[thm]{Axiom}
\newtheorem{set}[thm]{Axiom set}
\newtheorem{sch}[thm]{Axiom schema}
\newtheorem{princ}[thm]{Principle}
\newtheorem{algo}[thm]{Algorithm}
\newtheorem{tempie}[thm]{Template}
\newtheorem{ack}[thm]{Acknowledgement}
\newtheorem{specialcase}[thm]{Special Case}
\newtheorem{theme}[thm]{Theme}
\newtheorem{conj}[thm]{Conjecture}
\newtheorem*{tempo*}{Template}

\newcommand\be{\begin{equation}}
\newcommand\ee{\end{equation}} 
\def\bdefi{\begin{defi}\rm}
\def\edefi{\end{defi}}
\def\bnota{\begin{nota}\rm}
\def\enota{\end{nota}}
\def\FIVE{\Pi_{1}^{1}\text{-\textsf{CA}}_{0}}
\def\meta{\textup{\textup{meta}}}
\def\ATR{\textup{\textsf{ATR}}}
\def\ZFC{\textup{\textsf{ZFC}}}
\def\IST{\textup{\textsf{IST}}}
\def\MU{\textup{\textsf{MU}}}
\def\BET{\textup{\textsf{BET}}}
\def\ALP{\textup{\textsf{ALP}}}
\def\T{\mathcal{T}}
 \def\TT{\mathcal{TT}}
\def\STP{\textup{\textsf{STP}}}
\def\PA{\textup{PA}}
\def\FAN{\textup{\textsf{FAN}}}
\def\DNR{\textup{\textsf{DNR}}}
\def\RWWKL{\textup{\textsf{RWWKL}}}
\def\RWKL{\textup{\textsf{RWKL}}}
\def\H{\textup{\textsf{H}}}
\def\ef{\textup{\textsf{ef}}}
\def\ns{\textup{\textsf{ns}}}
\def\RCA{\textup{\textsf{RCA}}}
%\def\bennot{\textup{E-HA}^{\omega*}_{st}}
\def\({\textup{(}}
\def\){\textup{)}}
\def\WO{\textup{\textsf{WO}}}
\def\RCAo{\textup{\textsf{RCA}}_{0}^{\omega}}
\def\WKL{\textup{\textsf{WKL}}}
\def\PUC{\textup{\textsf{PUC}}}
\def\WWKL{\textup{\textsf{WWKL}}}
\def\bye{\end{document}}
%\def\rec{\textup{rec}}
%\def\sP{^{*}\mathcal  P}
\def\P{\textup{\textsf{P}}}
%\def\Pf{{\mathcal{P}_{\textup{fin}}}}
\def\N{{\mathbb  N}}
\def\Q{{\mathbb  Q}}
\def\R{{\mathbb  R}}
\def\L{{\mathfrak  L}}
\def\A{{\mathbb  A}}
\def\C{{\mathbb  C}}
\def\CC{{\mathfrak  C}}
\def\NN{{\mathfrak  N}}
\def\B{{\mathbb  B}}
\def\I{{\textsf{\textup{I}}}}
\def\D{{\mathbb  D}}
\def\E{{\mathcal  E}}
\def\FAN{\textup{\textsf{FAN}}}
\def\PUC{\textup{\textsf{PUC}}}
\def\WFAN{\textup{\textsf{WFAN}}}
\def\UFAN{\textup{\textsf{UFAN}}}
\def\MUC{\textup{\textsf{MUC}}}
\def\st{\textup{st}}
\def\di{\rightarrow}
\def\asa{\leftrightarrow}
\def\ACA{\textup{\textsf{ACA}}}
\def\paai{\Pi_{1}^{0}\textup{-\textsf{TRANS}}}
\def\Paai{\Pi_{1}^{1}\textup{-\textsf{TRANS}}}
\def\QFAC{\textup{\textsf{QF-AC}}}
\def\PRA{\textup{\textsf{PRA}}}
\def\WWF{\textup{\textsf{WWF}}}
\def\NUC{\textup{\textsf{NUC}}}
%\def\VCF{\textup{\textsf{VCF}}}
\def\LMP{\textup{\textsf{LMP}}}
%\def\RKL{\textup{\textsf{RKL}}}
\def\TJ{\textup{\textsf{TJ}}}
%\def\SJ{\textup{\textsf{SJ}}}
%\def\SHJ{\textup{\textsf{SHJ}}}
%\def\META{\textup{\textsf{META}}}
\def\SCF{\textup{\textsf{SCF}}}
\def\WCF{\textup{\textsf{WCF}}}
\def\HAC{\textup{\textsf{HAC}}}
\def\INT{\textup{\textsf{int}}}

\setcounter{tocdepth}{3}
\numberwithin{equation}{section}
\numberwithin{thm}{section}

\usepackage{comment}

\begin{document}
\title[Computability theory and Nonstandard Analysis]{Computability theory, Nonstandard Analysis, and their connections}
\author{Dag Normann}
\address{Department of Mathematics, The University 
of Oslo, P.O. Box 1053, Blindern N-0316 Oslo, Norway}
\email{dnormann@math.uio.no}
\author{Sam Sanders}
\address{Munich Center for Mathematical Philosophy, LMU Munich, Germany \& Department of Mathematics, Ghent University}
\email{sasander@me.com}

\begin{abstract}
In this paper we connect two seemingly unrelated topics, respectively in computability theory and Nonstandard Analysis.  % and the surprising connection between these topics.
In particular, we investigate the following:
\begin{enumerate}
 \renewcommand{\theenumi}{T.\arabic{enumi}}
\item We introduce the \emph{special fan functional} $\Theta$ and establish that it is easy to compute $\Theta$ in intuitionistic mathematics but hard to compute in classical mathematics.  
In particular, we show that the intuitionistic fan functional $\MUC$ can compute $\Theta$, but that the Turing jump functional $(\exists^{2})$ cannot (and the same for \emph{any} type two functional).
We show that the classical \emph{type three} functional $(\E_{2})$, which gives rise to full second-order arithmetic, can compute $\Theta$.  
Thus, \emph{first-order strength} and \emph{computational hardness} diverge significantly for the special fan functional \label{T1}
\item We study the nonstandard counterparts of the `Big Five' systems $\WKL_{0}$, $\ACA_{0}$, and $\FIVE$ of Reverse Mathematics, resp.\ the nonstandard compactness of Cantor space $\STP$ and the \emph{Transfer} axiom limited to $\Pi_{1}^{0}$-formulas $\paai$, and limited to $\Pi_{1}^{1}$-formulas $\Paai$.  While the Big Five of Reverse Mathematics are linearly ordered, 
and $\FIVE\di \ACA_{0}\di \WKL_{0}$ in particular, we show the non-implications $\paai\not\di \STP\not\!\leftarrow \Paai$ for the respective nonstandard counterparts.\label{T2}  % (and much stronger Transfer principles). 

\item We show that the results \eqref{T1} and \eqref{T2} are intimately connected.  In fact, the non-implications in \eqref{T2} are obtained \emph{directly} from the non-computability results in \eqref{T1}, and we show that non-computability results also follow from non-implications in Nonstandard Analysis.  
\end{enumerate}
\end{abstract}

\maketitle
\thispagestyle{empty}

%Template!  
%
\newpage
\tableofcontents

\section{Introduction}


\section{Background: internal set theory and Reverse Mathematics}\label{base}
In this section, we introduce Nelson's syntactic approach to Nonstandard Analysis \emph{internal set theory}, and it fragments based on Peano arithmetic from \cite{brie}. 
We also briefly sketch Friedman's foundational program \emph{Reverse Mathematics}.  
\subsection{Internal set theory and its fragments}\label{P}
In this section, we discuss Nelson's \emph{internal set theory}, first introduced in \cite{wownelly}, and its fragment $\P$ from \cite{brie}.  
The latter fragments are essential to our enterprise, especially by Theorem~\ref{consresultcor} below.  
\subsubsection{Internal set theory 101}\label{IIST}
In Nelson's \emph{syntactic} approach to Nonstandard Analysis (\cite{wownelly}), as opposed to Robinson's semantic one (\cite{robinson1}), a new predicate `st($x$)', read as `$x$ is standard' is added to the language of \textsf{ZFC}, the usual foundation of mathematics.  
The notations $(\forall^{\st}x)$ and $(\exists^{\st}y)$ are short for $(\forall x)(\st(x)\di \dots)$ and $(\exists y)(\st(y)\wedge \dots)$.  A formula is called \emph{internal} if it does not involve `st', and \emph{external} otherwise.   
The three external axioms \emph{Idealisation}, \emph{Standard Part}, and \emph{Transfer} govern the new predicate `st';  They are respectively defined\footnote{The superscript `fin' in \textsf{(I)} means that $x$ is finite, i.e.\ its number of elements are bounded by a natural number.} as:  
\begin{enumerate}
\item[\textsf{(I)}] $(\forall^{\st~\textup{fin}}x)(\exists y)(\forall z\in x)\varphi(z,y)\di (\exists y)(\forall^{\st}x)\varphi(x,y)$, for internal $\varphi$ with any (possibly nonstandard) parameters.  
\item[\textsf{(S)}] $(\forall^{\st} x)(\exists^{\st}y)(\forall^{\st}z)\big((z\in x\wedge \varphi(z))\asa z\in y\big)$, for any $\varphi$.
\item[\textsf{(T)}] $(\forall^{\st}t)\big[(\forall^{\st}x)\varphi(x, t)\di (\forall x)\varphi(x, t)\big]$, where $\varphi(x,t)$ is internal, and only has free variables $t, x$.  %  $t$ captures \emph{all} parameters of $\varphi$, and $t$ is standard.  
\end{enumerate}
The system \textsf{IST} is (the internal system) \textsf{ZFC} extended with the aforementioned external axioms;  
The former is a conservative extension of \textsf{ZFC} for the internal language, as proved in \cite{wownelly}.    % i.e.\ without `st'. 

\medskip

In \cite{brie}, the authors study G\"odel's system $\textsf{T}$ extended with special cases of the external axioms of \textsf{IST}.  
In particular, they consider the systems $\H$ and $\P$ which are conservative extensions of the (internal) logical systems \textsf{E-HA}$^{\omega}$ and $\textsf{E-PA}^{\omega}$, respectively \emph{Heyting and Peano arithmetic in all finite types and the axiom of extensionality}.       
We refer to \cite{kohlenbach3}*{\S3.3} for the exact definitions of the (mainstream in mathematical logic) systems \textsf{E-HA}$^{\omega}$ and $\textsf{E-PA}^{\omega}$.  
Furthermore, \textsf{E-PA}$^{\omega*}$ and $\textsf{E-HA}^{\omega*}$ are the definitional extensions of \textsf{E-PA}$^{\omega}$ and $\textsf{E-HA}^{\omega}$ with types for finite sequences, as in \cite{brie}*{\S2}.  For the former systems, we require some notation.  
\begin{nota}[Finite sequences]\label{skim}\rm
The systems $\textsf{E-PA}^{\omega*}$ and $\textsf{E-HA}^{\omega*}$ have a dedicated type for `finite sequences of objects of type $\rho$', namely $\rho^{*}$.  Since the usual coding of pairs of numbers goes through in both, we shall not always distinguish between $0$ and $0^{*}$.  % See e.g.\ the definition $(\GH)$ of the Gandy-Hyland functional in Section \ref{intro}.  
Similarly, we do not always distinguish between `$s^{\rho}$' and `$\langle s^{\rho}\rangle$', where the former is `the object $s$ of type $\rho$', and the latter is `the sequence of type $\rho^{*}$ with only element $s^{\rho}$'.  The empty sequence for the type $\rho^{*}$ is denoted by `$\langle \rangle_{\rho}$', usually with the typing omitted.  Furthermore, we denote by `$|s|=n$' the length of the finite sequence $s^{\rho^{*}}=\langle s_{0}^{\rho},s_{1}^{\rho},\dots,s_{n-1}^{\rho}\rangle$, where $|\langle\rangle|=0$, i.e.\ the empty sequence has length zero.
For sequences $s^{\rho^{*}}, t^{\rho^{*}}$, we denote by `$s*t$' the concatenation of $s$ and $t$, i.e.\ $(s*t)(i)=s(i)$ for $i<|s|$ and $(s*t)(j)=t(j-|s|)$ for $|s|\leq j< |s|+|t|$. For a sequence $s^{\rho^{*}}$, we define $\overline{s}N:=\langle s(0), s(1), \dots,  s(N)\rangle $ for $N^{0}<|s|$.  
For a sequence $\alpha^{0\di \rho}$, we also write $\overline{\alpha}N=\langle \alpha(0), \alpha(1),\dots, \alpha(N)\rangle$ for \emph{any} $N^{0}$.  By way of shorthand, $q^{\rho}\in Q^{\rho^{*}}$ abbreviates $(\exists i<|Q|)(Q(i)=_{\rho}q)$.  Finally, we shall use $\underline{x}, \underline{y},\underline{t}, \dots$ as short for tuples $x_{0}^{\sigma_{0}}, \dots x_{k}^{\sigma_{k}}$ of possibly different type $\sigma_{i}$.          
\end{nota}    
%
%\medskip
%
%In the next two sections, we introduce the systems $\P$ and $\H$ assuming familiarity with the higher-type framework (See e.g.\ \cite})  
\subsubsection{The classical system $\P$}\label{PIPI}
In this section, we introduce the system $\P$, a conservative extension of $\textsf{E-PA}^{\omega}$ with fragments of Nelson's $\IST$.  

\medskip

To this end, we first introduce the base system $\textsf{E-PA}_{\st}^{\omega*}$.  
We use the same definition as \cite{brie}*{Def.~6.1}, where \textsf{E-PA}$^{\omega*}$ is the definitional extension of \textsf{E-PA}$^{\omega}$ with types for finite sequences as in \cite{brie}*{\S2}.  
The set $\T^{*}$ is defined as the collection of all the constants in the language of $\textsf{E-PA}^{\omega*}$.    
\bdefi\label{debs}
The system $ \textsf{E-PA}^{\omega*}_{\st} $ is defined as $ \textsf{E-PA}^{\omega{*}} + \T^{*}_{\st} + \textsf{IA}^{\st}$, where $\T^{*}_{\st}$
consists of the following axiom schemas.
\begin{enumerate}
\item The schema\footnote{The language of $\textsf{E-PA}_{\st}^{\omega*}$ contains a symbol $\st_{\sigma}$ for each finite type $\sigma$, but the subscript is essentially always omitted.  Hence $\T^{*}_{\st}$ is an \emph{axiom schema} and not an axiom.\label{omit}} $\st(x)\wedge x=y\di\st(y)$,
\item The schema providing for each closed term $t\in \T^{*}$ the axiom $\st(t)$.
\item The schema $\st(f)\wedge \st(x)\di \st(f(x))$.
\end{enumerate}
The external induction axiom \textsf{IA}$^{\st}$ is as follows.  
\be\tag{\textsf{IA}$^{\st}$}
\Phi(0)\wedge(\forall^{\st}n^{0})(\Phi(n) \di\Phi(n+1))\di(\forall^{\st}n^{0})\Phi(n).     
\ee
\edefi
Secondly, we introduce some essential fragments of $\IST$ studied in \cite{brie}.  
\bdefi[External axioms of $\P$]~
%\be\tag{$\HAC_{\INT}$}
\begin{enumerate}
\item$\HAC_{\INT}$: For any internal formula $\varphi$, we have
\be\label{HACINT}
(\forall^{\st}x^{\rho})(\exists^{\st}y^{\tau})\varphi(x, y)\di \big(\exists^{\st}F^{\rho\di \tau^{*}}\big)(\forall^{\st}x^{\rho})(\exists y^{\tau}\in F(x))\varphi(x,y),
\ee
\item $\textsf{I}$: For any internal formula $\varphi$, we have
%\be\tag{\textsf{I}}
\[
(\forall^{\st} x^{\sigma^{*}})(\exists y^{\tau} )(\forall z^{\sigma}\in x)\varphi(z,y)\di (\exists y^{\tau})(\forall^{\st} x^{\sigma})\varphi(x,y), 
%?stx?? ?y? ?x??? x???x?,y????y? ?stx? ?(x,y).
\]
\item The system $\P$ is $\textsf{E-PA}_{\st}^{\omega*}+\textsf{I}+\HAC_{\INT}$.
\end{enumerate}
\end{defi}
Note that \textsf{I} and $\HAC_{\INT}$ are fragments of Nelson's axioms \emph{Idealisation} and \emph{Standard part}.  
By definition, $F$ in \eqref{HACINT} only provides a \emph{finite sequence} of witnesses to $(\exists^{\st}y)$, explaining its name \emph{Herbrandized Axiom of Choice}.   

\medskip

%Hence, the system $ \textsf{E-PA}^{\omega*}_{\st} $ is a trivial extension of \textsf{E-PA}$^{\omega^{*}}$.  
The system $\P$ is connected to $\textsf{E-PA}^{\omega}$ by the following theorem which expresses that we may obtain effective results as in \eqref{effewachten} from any theorem of Nonstandard Analysis which has the same form as in \eqref{bog}.  It is known that the scope of this corollary includes the Big Five systems of Reverse Mathematics (\cite{sambon}), the Reverse Mathematics zoo (\cite{samzoo, samzooII}), and higher-order computability theory (\cite{samGH}).  % (See Section \ref{RM}).  % while the RM zoo.     
\begin{thm}[Term extraction]\label{consresultcor}
%For internal $\psi$ and $\Phi(\underline{a})\equiv(\forall^{\st}\underline{x})(\exists^{\st}\underline{y})\psi(\underline{x},\underline{y}, \underline{a})$, we have $[\Phi(\underline{a})]^{S_{\st}}\equiv \Phi(\underline{a})$.  Hence, 
If $\Delta_{\INT}$ is a collection of internal formulas and $\psi$ is internal, and
\be\label{bog}
\P + \Delta_{\INT} \vdash (\forall^{\st}\underline{x})(\exists^{\st}\underline{y})\psi(\underline{x},\underline{y}, \underline{a}), 
\ee
then one can extract from the proof a sequence of closed terms $t$ in $\mathcal{T}^{*}$ such that
\be\label{effewachten}
\textup{\textsf{E-PA}}^{\omega*} + \Delta_{\INT} \vdash (\forall \underline{x})(\exists \underline{y}\in t(\underline{x}))\psi(\underline{x},\underline{y},\underline{a}).
\ee
%If for internal $\psi$ the formula $\Phi(\underline{a})\equiv(\forall^{\st}\underline{x})(\exists^{\st}\underline{y})\psi(\underline{x},\underline{y}, \underline{a})$ satisfies \eqref{antecedn}, then 
%$(\forall \underline{x})(\exists \underline{y}\in t(\underline{x}))\psi(\underline{x},\underline{y},\underline{a})$ is proved in the corresponding formula \eqref{consequalty}.  
\end{thm}
\begin{proof}
See \cite{samGH}*{\S2} or \cite{sambon}*{\S2}.
\end{proof}
Curiously, the previous theorem is neither explicitly listed or proved in \cite{brie}.
For the rest of this paper, the notion `normal form' shall refer to a formula as in \eqref{bog}, i.e.\ of the form $(\forall^{\st}x)(\exists^{\st}y)\varphi(x,y)$ for $\varphi$ internal.  

\medskip

Finally, the previous theorems do not really depend on the presence of full Peano arithmetic.  
We shall study the following subsystems.   
\bdefi~
\begin{enumerate}
\item Let \textsf{E-PRA}$^{\omega}$ be the system defined in \cite{kohlenbach2}*{\S2} and let \textsf{E-PRA}$^{\omega*}$ 
be its definitional extension with types for finite sequences as in \cite{brie}*{\S2}. 
\item $(\QFAC^{\rho, \tau})$ For every quantifier-free internal formula $\varphi(x,y)$, we have
\be\label{keuze}
(\forall x^{\rho})(\exists y^{\tau})\varphi(x,y) \di (\exists F^{\rho\di \tau})(\forall x^{\rho})\varphi(x,F(x))
\ee
\item The system $\RCAo$ is $\textsf{E-PRA}^{\omega}+\QFAC^{1,0}$.  
\end{enumerate}
\edefi
The system $\RCAo$ is Kohlenbach's `base theory of higher-order Reverse Mathematics' as introduced in \cite{kohlenbach2}*{\S2}.  
We permit ourselves a slight abuse of notation by also referring to the system $\textsf{E-PRA}^{\omega*}+\QFAC^{1,0}$ as $\RCAo$.
\begin{cor}\label{consresultcor2}
The previous theorem and corollary go through for $\P$ and $\textsf{\textup{E-PA}}^{\omega*}$ replaced by $\P_{0}\equiv \textsf{\textup{E-PRA}}^{\omega*}+\T_{\st}^{*} +\HAC_{\INT} +\textsf{\textup{I}}+\QFAC^{1,0}$ and $\RCAo$.  
\end{cor}
\begin{proof}
The proof of \cite{brie}*{Theorem 7.7} goes through for any fragment of \textsf{E-PA}$^{\omega{*}}$ which includes \textsf{EFA}, sometimes also called $\textsf{I}\Delta_{0}+\textsf{EXP}$.  
In particular, the exponential function is (all what is) required to `easily' manipulate finite sequences.    
\end{proof}
Finally, we note that Ferreira and Gaspar present a system similar to $\P$ in \cite{fega}, which however is less suitable for our purposes.    

\subsubsection{Notations and conventions}
We introduce notations and conventions for $\P$.  

\medskip

First of all, we mostly use the same notations as in \cite{brie}.  % some of which we repeat here.  
\begin{rem}[Notations]\label{notawin}\rm
We write $(\forall^{\st}x^{\tau})\Phi(x^{\tau})$ and $(\exists^{\st}x^{\sigma})\Psi(x^{\sigma})$ as short for 
$(\forall x^{\tau})\big[\st(x^{\tau})\di \Phi(x^{\tau})\big]$ and $(\exists^{\st}x^{\sigma})\big[\st(x^{\sigma})\wedge \Psi(x^{\sigma})\big]$.     
We also write $(\forall x^{0}\in \Omega)\Phi(x^{0})$ and $(\exists x^{0}\in \Omega)\Psi(x^{0})$ as short for 
$(\forall x^{0})\big[\neg\st(x^{0})\di \Phi(x^{0})\big]$ and $(\exists x^{0})\big[\neg\st(x^{0})\wedge \Psi(x^{0})\big]$.  
%Furthermore, if $\neg\st(x^{0})$ (resp.\ $\st(x^{0})$), we also say that $x^{0}$ is `infinite' (resp.\ finite) and write `$x^{0}\in \Omega$'.  %In keeping with the internal 
Finally, a formula $A$ is `internal' if it does not involve $\st$, and $A^{\st}$ is defined from $A$ by appending `st' to all quantifiers (except bounded number quantifiers).    
\end{rem}
\medskip

Secondly, we use the usual extensional notion of equality. 
%None of our results depend crucially on this notion, i.e.\ we would obtain the same results for non-extensional notions of equality by dropping the axiom \eqref{EXT} introduced now.    
\begin{rem}[Equality]\label{equ}\rm
The system $\textsf{E-PA}^{\omega*}$ includes equality between natural numbers `$=_{0}$' as a primitive.  Equality `$=_{\tau}$' and inequality $\leq_{\tau}$ for $x^{\tau},y^{\tau}$ is:
\be\label{aparth}
[x=_{\tau}y] \equiv (\forall z_{1}^{\tau_{1}}\dots z_{k}^{\tau_{k}})[xz_{1}\dots z_{k}=_{0}yz_{1}\dots z_{k}],
\ee
\be\label{aparth1}
[x\leq_{\tau}y] \equiv (\forall z_{1}^{\tau_{1}}\dots z_{k}^{\tau_{k}})[xz_{1}\dots z_{k}\leq_{0}yz_{1}\dots z_{k}],
\ee
if the type $\tau$ is composed as $\tau\equiv(\tau_{1}\di \dots\di \tau_{k}\di 0)$.
In the spirit of Nonstandard Analysis, we define `approximate equality $\approx_{\tau}$' as follows:
\be\label{aparth2}
[x\approx_{\tau}y] \equiv (\forall^{\st} z_{1}^{\tau_{1}}\dots z_{k}^{\tau_{k}})[xz_{1}\dots z_{k}=_{0}yz_{1}\dots z_{k}]
\ee
with the type $\tau$ as above.  
%As noted in \cite{brie}*{p.\ 1973}, weak nonstandard systems can include the axiom of extensionality for $=_{\tau}$, but not for $\approx_{\tau}$.
All the above systems include the \emph{axiom of extensionality} for all $\varphi^{\rho\di \tau}$ as follows:
\be\label{EXT}\tag{\textsf{E}}  
(\forall  x^{\rho},y^{\rho}) \big[x=_{\rho} y \di \varphi(x)=_{\tau}\varphi(y)   \big].
\ee
However, as noted in \cite{brie}*{p.\ 1973}, the so-called axiom of \emph{standard} extensionality \eqref{EXT}$^{\st}$ is problematic and cannot be included in $\P$.  
%Finally, a functional $\Xi^{ 1\di 0}$ is called an \emph{extensionality functional} for $\varphi^{1\di 1}$ if 
%\be\label{turki}
%(\forall k, f^{1}, g^{1})\big[ \overline{f}\Xi(f,g, k)=_{0}\overline{g}\Xi(f,g,k) \di \overline{\varphi(f)}k=_{0}\overline{\varphi(g)}k \big],  
%\ee
%i.e.\ $\Xi$ witnesses \eqref{EXT} for $\varphi$.  
%As will become clear in Section \ref{X}, standard extensionality is translated by our template $\CI$ into the existence of an extensionality functional, and the latter amounts to no more than an unbounded search.   % (or $\QFAC^{1,0}$).     
\end{rem}
Thirdly, the system $\P$ proves overspill and underspill, which are quite useful principles.
\begin{thm}\label{doppi}
The systems $\P$ and $\P_{0}$ prove \emph{overspill}, i.e.\
\be\tag{\textsf{OS}}
(\forall^{\st}x^{\rho})\varphi(x)\di (\exists y^{\rho})\big[\neg\st(y)\wedge \varphi(y)  \big],
\ee
for any internal formula $\varphi$.
\end{thm}
\begin{proof}
See \cite{brie}*{Prop.\ 3.3}.  
\end{proof}
Fourth, we consider the following remark on how $\HAC_{\INT}$ and $\textsf{I}$ are used.  
\begin{rem}[Using $\HAC_{\INT}$ and $\textsf{I}$]\label{simply}\rm
By definition, $\HAC_{\INT}$ produces a functional $F^{\sigma\di \tau^{*}}$ which outputs a \emph{finite sequence} of witnesses.  
However, $\HAC_{\INT}$ provides an actual \emph{witnessing functional} assuming (i) $\tau=0$ in $\HAC_{\INT}$ and (ii) the formula $\varphi$ from $\HAC_{\INT}$ is `sufficiently monotone' as in: 
$(\forall^{\st} x^{\sigma},n^{0},m^{0})\big([n\leq_{0}m \wedge\varphi(x,n)] \di \varphi(x,m)\big)$.    
Indeed, in this case one simply defines $G^{\sigma+1}$ by $G(x^{\sigma}):=\max_{i<|F(x)|}F(x)(i)$ which satisfies $(\forall^{\st}x^{\sigma})\varphi(x, G(x))$.  
% Indeed, consider \eqref{tochie} and note that the internal formula in the latter is monotone as above.  Taking the maximum of $t$ from \eqref{tochie} as $u(g, k'):=\max_{i<|t(g,k')|}t(g, k')(i)$, one can drop the quantifier `$(\exists N'\in t(g,k'))$' in \eqref{tochie} to obtain \eqref{tochie2} and \eqref{crux}.  
To save space in proofs, we will sometimes skip the (obvious) step involving the maximum of finite sequences, when applying $\HAC_{\INT}$.  %$\textsf{I}$ and term extraction. 
We assume the same convention for terms obtained from Theorem \ref{consresultcor}, and applications of the contraposition of idealisation \textsf{I}.  
\end{rem}


\subsection{Introducing Reverse Mathematics}\label{RM}
Reverse Mathematics (RM) is a program in the foundations of mathematics initiated around 1975 by Friedman (\cites{fried,fried2}) and developed extensively by Simpson (\cite{simpson2, simpson1}) and others.  We refer to \cite{simpson2} for an introduction to RM; we do sketch some of its aspects essential to this paper.  

\medskip
  
The aim of RM is to find the axioms necessary to prove a statement of \emph{ordinary} mathematics, i.e.\ dealing with countable or separable objects.   
The classical\footnote{In \emph{Constructive Reverse Mathematics} (\cite{ishi1}), the base theory is based on intuitionistic logic.} base theory $\RCA_{0}$ of `computable\footnote{The system $\RCA_{0}$ consists of induction $I\Sigma_{1}$, and the {\bf r}ecursive {\bf c}omprehension {\bf a}xiom $\Delta_{1}^{0}$-CA.} mathematics' is always assumed.  
Thus, the aim is:  
\begin{quote}
\emph{The aim of \emph{RM} is to find the minimal axioms $A$ such that $\RCA_{0}$ proves $ [A\di T]$ for statements $T$ of ordinary mathematics.}
\end{quote}
Surprisingly, once the minimal axioms $A$ have been found, we almost always also have $\RCA_{0}\vdash [A\asa T]$, i.e.\ not only can we derive the theorem $T$ from the axioms $A$ (the `usual' way of doing mathematics), we can also derive the axiom $A$ from the theorem $T$ (the `reverse' way of doing mathematics).  In light of the latter, the field was baptised `Reverse Mathematics'.    

\medskip

Perhaps even more surprisingly, in the majority\footnote{Exceptions are classified in the so-called Reverse Mathematics zoo (\cite{damirzoo}).
} 
of cases for a statement $T$ of ordinary mathematics, either $T$ is provable in $\RCA_{0}$, or the latter proves $T\asa A_{i}$, where $A_{i}$ is one of the logical systems $\WKL_{0}, \ACA_{0},$ $ \ATR_{0}$ or $\FIVE$.  The latter together with $\RCA_{0}$ form the `Big Five' and the aforementioned observation that most mathematical theorems fall into one of the Big Five categories, is called the \emph{Big Five phenomenon} (\cite{montahue}*{p.~432}).  
Furthermore, each of the Big Five has a natural formulation in terms of (Turing) computability (See e.g.\ \cite{simpson2}*{I.3.4, I.5.4, I.7.5}).
As noted by Simpson in \cite{simpson2}*{I.12}, each of the Big Five also corresponds (sometimes loosely) to a foundational program in mathematics.  

\medskip

Now, the logical framework for RM is \emph{second-order arithmetic}, i.e.\ only natural numbers and sets thereof are available.  
For this reason higher-order objects such as continuous real functions and topologies are not available directly, and are represented by so-called codes (See e.g.\ \cite{simpson2}*{II.6.1} and \cite{mummy}).  Kohlenbach has introduced \emph{higher-order} RM and the associated base theory $\RCAo$ where the language includes all finite types; we refer to \cite{kohlenbach2}*{\S2} for the definition of the latter system.  

\medskip

Finally, we consider an interesting observation regarding the Big Five systems of Reverse Mathematics, namely that these five systems satisfy the strict implications:
\be\label{linord}
\FIVE\di \ATR_{0}\di \ACA_{0}\di\WKL_{0}\di \RCA_{0}.
\ee
By contrast, there are many incomparable logical statements in second-order arithmetic.  For instance, a regular plethora of such statements may be found in the \emph{Reverse Mathematics zoo} in \cite{damirzoo}.  The latter is a collection of theorems which fall outside of the Big Five classification of RM.  

\section{Main results}
In this section, we prove the results sketched in the introduction.  

\subsection{Computing the special fan functional}\label{prim}
In this section, we study the relationship between the new \emph{special fan functional} and mainstream functionals like the \emph{Turing jump functional}.
As a main result, we show that the latter (and in fact any type two functional) cannot compute (in the sense of Kleene's S1-S9 from \cite{longmann}*{\S8}) the special fan functional.    

\medskip

As to its provenance, the special fan functional was first introduced in \cite{samGH}*{\S3} in the study of the Gandy-Hyland functional.  
The special fan functional is an object of classical mathematics in that it can be defined in a (relatively strong) fragment of set theory by Theorem \ref{import3} in Section~\ref{tokie}.  
Furthermore, the special fan functional  may be derived from the \emph{intuitionistic} fan functional, as shown in Section \ref{indie}.  The latter result shows that the existence of the special fan functional has quite weak first-order strength (in contrast to its computational strength).       %hardness?

\subsubsection{The special and intuitionistic fan functionals}\label{indie}
In this section, we introduce the functionals from the title and show that the latter computes the former via a term from G\"odel's $\textsf{T}$.  
In particular, the name `special fan functional' derives from this relative computability result.   
%{enumerate}
%\item  A special fan functional may be computed (via a term in G\"odel's \textsf{T}) from the \emph{intuitionistic fan functional} $\Omega^{3}$ as in Definition \ref{into}.
%\item Any special fan functional computes (via a term in G\"odel's \textsf{T}) the effective version of the fan theorem.
%\end{enumerate}  

\medskip

First of all, we define the special fan functional.  We reserve the variables $S^{1}, T^{1}, U^{1}$ for trees and denote by `$T^{1}\leq_{1}1$' that $T$ is a binary tree.  Recall that $1^{*}$ is the type of finite sequences of type $1$ as in Notation \ref{skim}. 
\bdefi[Special fan functional]
We define $\SCF(\Theta)$ as follows for $\Theta^{(2\di (0\times 1^{*}))}$:
\[
(\forall g^{2}, T^{1}\leq_{1}1)\big[(\forall \alpha \in \Theta(g)(2))  (\overline{\alpha}g(\alpha)\not\in T)
\di(\forall \beta\leq_{1}1)(\exists i\leq \Theta(g)(1))(\overline{\beta}i\not\in T) \big]. 
\]
Any functional $\Theta$ satisfying $\SCF(\Theta)$ is referred to as a \emph{special fan functional}.
\edefi
Note that there is \emph{no unique} special fan functional, i.e.\ it is in principle incorrect to make statements about `the' special fan functional. 

\medskip

Secondly, we define the \emph{intuitionistic fan functional} $\Omega$ as in \cite{kohlenbach2}*{\S3} and \cite{troelstra1}*{2.6.6}.  
\bdefi[Intuitionistic fan functional]\label{into}
\be\tag{$\MUC(\Omega)$}
(\forall Y^{2}) (\forall f, g\leq_{1}1)(\overline{f}\Omega(Y)=\overline{g}\Omega(Y)\notag \di Y(f)=Y(g)),   % \label{lukl3}\tag{$\textsf{\textup{MUC}}(\Phi)$}$
\ee
\edefi
As to the logical strength of $(\exists \Omega^{3})\MUC(\Omega)$, the latter yields a conservative extension of $\WKL_{0}$ by the following theorem, where `$\RCA_{0}^{2}$' is just the base theory $\RCA_{0}$ formulated with function variables rather than set variables (See \cite{kohlenbach2}*{\S2}).  
%, where all systems are introduced in \cite{kohlenbach2}*{\S2}.  
%Recall that $\WKL$ is \emph{weak K\"onig's lemma}, i.e.\ the statement that every infinite binary tree has a path.  A proof of the following theorem may be found at the end of this section.      
\begin{thm}\label{proto}
The system $\RCA_{0}^{\omega}+(\exists \Omega^{3})\MUC(\Omega)$ is a conservative extension of $\RCA_{0}^{2}+\WKL$ \(for the second-order language of the latter\).  
\end{thm}
\begin{proof}
A very rudimentary sketch of a proof is provided in \cite{kohlenbach2}*{\S3}. 
A detailed proof is provided in Theorem \ref{protofinal} of the Appendix.  
\end{proof}
Recall that the fan theorem $\FAN$  is the classical contraposition of $\WKL$. 
\be\tag{$\FAN$}
(\forall T \leq_{1}1)\big[ (\forall \beta\leq_{1}1)(\exists m)(\overline{\beta}m\not \in T)\di (\exists k^{0})(\forall \beta\leq_{1}1)(\exists i\leq k)(\overline{\beta}i\not\in T) \big]. 
\ee
We also introduce the `effective version' of the fan theorem as follows.
\bdefi[Effective fan theorem]
\be\tag{$\FAN_{\ef}(h)$}
(\forall T^{1}\leq_{1}1, g^{2})\big[ (\forall \alpha\leq_{1}1)(\overline{\alpha}g(\alpha)\not\in T)\di (\forall \beta\leq_{1}1)(\overline{\beta}h(g, T)\not\in T)   \big].
\ee
\edefi
Clearly, the existence of $h$ as in the effective fan theorem implies $\FAN$ in $\RCAo$.  
Furthermore, with a further minimum of the axiom of choice $\QFAC^{2,1}$, the latter also follows form the former.  
We have the following theorem. 
%The following theorem was announced earlier.   
%\bdefi[Intuitionistic fan functional]
%\be\tag{$\MUC(\Omega)$}
%(\forall Y^{2}) (\forall f, g\leq_{1}1)(\overline{f}\Omega(Y)=\overline{g}\Omega(Y)\notag \di Y(f)=Y(g)),   % \label{lukl3}\tag{$\textsf{\textup{MUC}}(\Phi)$}$
%\ee
%\edefi
\begin{thm}\label{kinkel}
There are terms $s,t$ such that $\textsf{\textup{E-PA}}^{\omega}$ proves:
\be\label{ikkeltje}
(\forall \Omega^{3})(\MUC(\Omega)\di \SCF(t(\Omega))) \wedge (\forall \Theta)(\SCF(\Theta)\di \FAN_{\ef}(s(\Theta))).
\ee
\end{thm}
\begin{proof}
The second part of \eqref{ikkeltje} is immediate.  
For the first part, let $\Omega$ be as in $\MUC(\Omega)$ and note that $\Theta(g)$ as in $\SCF(\Theta)$ has to provide a natural number and a finite sequence of binary sequences.
The number $\Theta(g)(1)$ is defined as $\max_{|\sigma|=\Omega(g)\wedge \sigma\leq_{0^{*}}1}g(\sigma*00\dots)$ and the finite sequence of binary sequences $\Theta(g)(2)$
consists of all $\tau*00\dots$ where $|\tau|=\Theta(g)(1)\wedge \tau\leq_{0^{*}}1$.  
We now claim that for all $g^{2}$ and $T^{1}\leq_{1}1$:
\be\label{difffff}
 (\forall \beta\leq_{1}1)(\beta \in \Theta(g)(2)\di \overline{\beta}{g}(\beta)\not \in T)\di (\forall \gamma\leq_{1}1)(\exists i\leq \Theta(g)(1))(\overline{\gamma}i\not \in T).
\ee
%where the forward implication is trivial and the reverse implication follows from the definition of $\Theta$.   %as for all $\beta\leq_{1}1$, we have $\tilde{g}(\beta)=\tilde{g}(\overline{\beta}\Theta(g)(1)*00)\not \in T$.  
Indeed, suppose the antecendent of \eqref{difffff} holds.  Now take $\gamma_{0}\leq_{1}1$, and note that $\beta_{0}=\overline{\gamma_{0}}\Theta(g)(1)*00\dots \in \Theta(g)(2)$, implying 
$\overline{\beta_{0}}{g}(\beta_{0})\not \in T$.  But $g(\alpha)\leq \Theta(g)(1)$ for all $\alpha\leq_{1}1$, by the definition of $\Omega$, implying that $\overline{\gamma_{0}}{g}(\beta_{0})=\overline{\beta_{0}}{g}(\beta_{0})\not \in T$ by the definition of $\beta_{0}$, and the consequent of \eqref{difffff} follows.    
%Hence, we have used the fan functional to define a functional $\Theta$ which satisfies $\SCF(\Theta)$.  
% \eqref{difffff} for all $g^{2}$ and all $T\leq_{1}1$, implying that $(\forall g^{2})B(g, \Theta(g))$ as required.  
\end{proof}
As it happens, the first part of Theorem \ref{kinkel} was first proved \emph{indirectly} in \cite{samGH}*{\S3} by applying Theorem~\ref{consresultcor} to the normal form of $\NUC\di \STP$, where 
\be\tag{$\NUC$}
(\forall^{\st}Y^{2})(\forall f^{1}, g^{1}\leq_{1}1)(f\approx_{1}g\di Y(f)=_{0}Y(g)), 
\ee
i.e.\ the statement that every type two functional is nonstandard uniformly continuous on Cantor space in light of Notation \ref{equ}.  

\medskip


Furthermore, the `classical' fan functional is obtained from the intuitionistic one by restricting $Y^{2}$ in $\MUC(\Omega)$ to $Y^{2}\in C$, i.e.\ continuous as follows:
\[
Y^{2}\in C\equiv (\forall f^{1})(\exists N^{0})(\forall g^{1})(\overline{f}N=\overline{g}N\di Y(f)=Y(g)).  
\]
By combining \cite{kohlenbach4}*{Prop.\ 4.4 and 4.7}, the Turing jump functional can compute the classical fan functional (over Kohlenbach's system $\RCAo$ from \cite{kohlenbach2}*{\S2}).  

\medskip

In light of the previous observations regarding the classical and intuitionistic fan functionals, a special fan functional appears to be a rather weak object.     
Looks can be deceiving, as we establish in Theorem \ref{import} in the next section that the Turing jump functional (and in fact any type two functional) cannot compute any special fan functional.
We finish this section with a remark on the definition of the special fan functional.
\begin{rem}\rm
$\Phi(T, g)$ in INT and CLASS. 
\end{rem}

\subsubsection{The special fan functional and comprehension functionals}\label{tokie}
In this section, we study the relationship between the special fan functional and comprehension functionals.  
In particular, we show that the former cannot be computed by the \emph{Turing jump functional} (and any type two functional) defined as follows.
\bdefi[Turing jump functional]\label{TJ}
\be\tag{$\TJ(\varphi)$}
(\forall f^{1})\big[(\exists n)(f(n)=0)\asa \varphi(f)=0  \big].
\ee
We let $(\exists^2)$ stand for $(\exists \varphi^{2})\TJ(\varphi)$, and call $\varphi$ the `Turing jump functional'. 
\edefi
We make our notion of `computability' precise as follows.  
\begin{enumerate}
\item We adopt $\ZFC$ set theory as the official metatheory for all results, unless explicitly stated otherwise.
\item We adopt Kleene's notion of \emph{higher-order computation} as given by his nine clauses S1-S9 (See \cite{longmann}*{\S8}) as our official notion of `computable'.
\end{enumerate}
%The Turing jump functional is then defined as follows.    % and their basic properties.  
With these conventions in place, we can prove the following theorem.
\begin{thm}\label{import}
Any functional $\Theta^{3}$ as in $\SCF(\Theta)$ is not computable in $(\exists^{2})$.   
\end{thm}
\begin{proof}
Assume that $\Theta$ as in $\SCF(\Theta)$ is computable in $\varphi$ as in $\TJ(\varphi)$. 
Let $h^{2}$ be any partial functional computable in $\varphi$ as in $\TJ(\varphi)$ and total on the class of hyperarithmetical functions, and let $g^{2}$ be the total extension of $h$.
Then $\Theta$ applied to $g$ will yield a hyperarithmetical finite sequence $\Theta(g)(1)$.

\medskip

We now define $h_{0}(\alpha)$, using Gandy selection, using the `least' number $e$ such that $e$ is an index for $\alpha$ as a hyperarithmetical function in some fixed canonical indexing of the hyperarithmetical sets. By `least' we mean `of minimal ordering rank', and then of minimal numerical value among those.
In particular, define $h_{0}(\alpha) = e + 2$ for the aforementioned $e$, and let $g_{0}$ be the associated extension discussed above.  Then $\Theta(g_{0})(1)$ consists of a  finite list $\alpha_1 , . . . , \alpha_k$ of hyperarithmetical functions, and the neighbourhoods determined by the $\overline{ \alpha_i}(g(\alpha_i))$ are not of measure 1, so they do not cover the Cantor space.

\medskip

However, then there is a non-well founded binary tree $T_{0}$ such that $ \overline{\alpha_i}(g_{0}(\alpha_i) \not \in T_{0}$ for all $i = 1 , . . . , k$, but there is no possible value for $\Theta(g_{0})(2)$.
\end{proof}
\begin{cor}\label{import2}
Let $\varphi^{2}$ be any type two functional.   
Any functional $\Theta^{3}$ as in $\SCF(\Theta)$ is not computable in $\varphi^{2}$.   
\end{cor}
\begin{proof}
X
\end{proof}
We now list some well-known type two functionals which will also be used below.  
%Clearly, the previous theorem also holds for $(\mu^{2})$ as follows:
\emph{Feferman's search operator} $(\mu^{2})$ (See e.g.\ \cite{avi2}*{\S8}) is equivalent to $(\exists^{2})$ over Kohlenbach's system $\RCAo$ by \cite{kooltje}*{\S3}:.
\be\tag{$\mu^{2}$}
(\exists \mu^{2})\big[(\forall f^{1})\big( (\exists n^{0})(f(n)=0)\di f(\mu(f))=0   \big)  \big],
\ee
and is the functional version of $\ACA_{0}$.  The \emph{Suslin functional} and $(\mu_{1})$ (See \cite{avi1}*{\S8.4.1}, \cite{kohlenbach2}*{\S1}, and \cite{yamayamaharehare}*{\S3}) are the functional versions of $\FIVE$, and defined as: 
\be\tag{$S^{2}$}
(\exists S^{2})(\forall f^{1})\big[  (\exists g^{1})(\forall x^{0})(f(\overline{g}n)=0)\asa S(f)=0  \big].
\ee
\be\tag{$\mu_{1}$}
(\exists \mu_{1}^{1\di 1})(\forall f^{1})\big[  (\exists g^{1})(\forall x^{0})(f(\overline{g}n)=0)\di (\forall x^{0})(f(\overline{\mu_{1}(f)}n)=0)  \big].
\ee
%
%\medskip
%
%The following theorem sharpens the above result.
%\begin{thm}\label{import2}
%Any functional $\Theta$ as in $\SCF(\Theta)$ is not computable in the Suslin functional:
%\be\tag{$S^{2}$}
%(\exists S^{2})(\forall f^{1})\big[  (\exists g^{1})(\forall x^{0})(f(\overline{g}n)=0)\asa S(f)=0  \big].
%\ee
%\end{thm}
%\begin{proof}
%X
%\end{proof}
On the other hand, full second-order arithmetic suffices to compute a special fan functional, as we show now.
\begin{thm}\label{import3}
A functional $\Theta^{3}$ as in $\SCF(\Theta)$ can be computed from $\xi$ as in $(\mathcal{E}_{2})$: % as follows:
\be\tag{$\mathcal{E}_{2}$}\label{hah}
(\exists \xi^{3})(\forall Y^{2})\big[  (\exists f^{1})(Y(f)=0)\asa \xi(Y)=0  \big].
\ee
\end{thm}
\begin{proof}
We first prove the existence of a functional $\Theta$ such that $\SCF(\Theta)$ in classical set theory without choice $\textsf{ZF}$.  
We then show how the construction can be realised as an algorithm relative to $\xi$ as in $(\E_{2})$.   %  quantification over $\N^\N$.
 
 %\subsection*{Notation}
 Let $C$ be the Cantor space $\{0,1\}^\N$  with the lexicographical ordering. If $\sigma$ is a binary finite sequence, we let $C_\sigma$ be the set of binary extensions of $\sigma$ (
 in $C$).
 \newline
 We let $f,g$ with indices vary over $C$ and we let $\alpha$, $\beta$ etc. vary over the countable ordinals. We let $F$ be a fixed total functional of type 2, and our aim is to define $\Theta(F)$.
 
%\subsection*{The construction}
 By recursion on $\alpha$ we will define an increasing sequence $\{f_\alpha\}_{\alpha < \aleph_1}$ from $C$.
 We will let $$I(\alpha) = \bigcup_{\beta \leq \alpha}C_{\bar f_\beta(F(f_\beta))}$$ and we will let 
 
$$I(< \alpha) = \bigcup_{\beta < \alpha}C_{\bar f_\beta(F(f_\beta))}.$$
We let $f_0 = \lambda x.0$.
%
\newline Let $\alpha > 0$. 
\begin{itemize}
\item If $\lambda x.1 \in I( < \alpha)$, let $f_\alpha = f_\beta$ for the first $\beta$ such that $\lambda x.1 \in C_{\bar f_\beta(F(f_\beta))}$.
\item If not, let $f_\alpha$ be the least element not in $I(< \alpha)$.
\end{itemize}
By construction, the sequence of $f_\alpha$'s will be strictly increasing until we capture $\lambda x.1$, which thus must happen after a countable number of steps.
\vspace{2mm}
 \newline
 Clearly, the least $\alpha$ such that $f \in I(\alpha)$ must be a successor ordinal for each $f$. Let $\alpha_0$ be this ordinal for $f = \lambda x.1$, and let $g_0$ be the greatest strict lower bound of $C_{\bar f_{\alpha_0}(F(f_{f_{\alpha_0}}))}.$
 \newline
 Let $\alpha_1$ be this ordinal for $f = g_0$ and let $g_1$ be the greatest strict lower bound of $C_{\bar f_{\alpha_1}(F(f_{f_{\alpha_1}}))}.$
 \newline
 Continue this process, defining a decreasing sequence $\alpha_0,\alpha_1, \cdots$  until $\lambda x.0$ is captured, and we have a finite cover of $C$ of neighborhoods of the form $C_{\bar f_{\alpha_i}(F(f_{\alpha_i}))}$ for $i \leq n$  for some $n$.
 \newline
 We then let $\Theta(F)$ have the pair of $\{f_{\alpha_i}\mid i  \leq n\}$ and $\max\{F(f_{\alpha_i})\mid i \leq n\}$ as value.
 
 \medskip
 
% \subsubsection*{Computing $\Theta$ from $^3E$}
 We need far less that $^3E$ to capture this construction, but it may be difficult to isolate a simpler functional in which $\Theta$ is computable.
 \newline
 Using $^3E$ I would proceed as follows:
 \begin{enumerate}
 \item Let $WO$ be a standard $\Pi^1_1$ set of codes for the countable ordinals.
 \item From $F$ and $^3E$ we can compute the set of ordinals of order type $\alpha_0$ of the construction.
 \item For each of these codes, we can compute from $F$ the sequence $\{f_\beta\}_{\beta \leq \alpha_0}$, and for each of these codes, we can compute from $F$ the backtracking to $\alpha_1 , \ldots , \alpha_n$.
 \item Since the construction only depends on the ordinals, it does not matter which code for an ordinal we use, we end up with the same value for $\Theta(F)$.
 \item Extracting this common value is computable in $^3E$.
 \end{enumerate}
\end{proof}
Finally, we show that the special fan functional is not special in the sense that there are a number of functionals with closely related `computational' properties (although the nonstandard provenance of the former arguably remains `special').  
By way of an example, we consider the following functionals, where we write `$(\forall \alpha^{1} \in \gamma)\varphi(\alpha)$' instead of `$(\forall n^{0})\varphi(\gamma(n))$' for $\gamma^{0\di 1}$. 
\bdefi
Define $\ALP(\chi)$ for $\chi^{2\di 1^{*}}$ as follows:
\[\textstyle
(\forall g^{2})(\exists k^{0})(\forall T^{1}\leq_{1}1)\big[(\forall \alpha \in \chi(g))  (\overline{\alpha}g(\alpha)\not\in T)
\di (\forall \alpha^{1}\leq_{1}1)(\exists i\leq k)(\overline{\alpha}i\not\in T)\big].
\]
%where `$T^{1}\leq_{1}1$' is meant to convey that $T$ is a binary tree, and where $1^{*}$ is the type of finite sequences of type $1$. 
%Any functional $\zeta$ as in $\WWF(\zeta)$ is referred to as a \emph{weak weak special fan functional}.
\edefi
\bdefi
Define $\BET(\iota)$ for $\iota^{(2\di (0\times (0\di 1)))}$ as follows:
\[\textstyle
(\forall g^{2}, T^{1}\leq_{1}1)\big[(\forall \alpha \in \iota(g)(2))  (\overline{\alpha}g(\alpha)\not\in T)
\di  (\forall \alpha^{1}\leq_{1}1)(\exists i\leq \iota(g)(1))(\overline{\alpha}i\not\in T)\big].
\]
%where `$T^{1}\leq_{1}1$' is meant to convey that $T$ is a binary tree, and where $1^{*}$ is the type of finite sequences of type $1$. 
%Any functional $\zeta$ as in $\WWF(\zeta)$ is referred to as a \emph{weak weak special fan functional}.
\edefi
By the following theorem, the special fan functional is `sandwiched' between these new functionals and the latter plus Feferman's search operator.  
\begin{thm}
There are terms $s, t, u, v$ such that $\textsf{\textup{E-PA}}^{\omega*}$ proves that\\
$(\forall \Theta)( \ALP(u(\Theta))\leftarrow  \SCF(\Theta)\di \BET(t(\Theta)))$ and that
\[
(\forall \zeta, \mu)\big(    (\BET(\zeta)\wedge \MU(\mu))\di \SCF(s(\zeta, \mu)) \leftarrow (\ALP(\zeta)\wedge\MU(\mu))  \big).  % and the same for $\chi$ as in $\ALP(\chi)$. 
\]
\end{thm}
\begin{proof}
For the first part of the proof, define $t(\Theta)(g)(1):=\Theta(g)(1)$ and define $t(\Theta)(g)(2)(k)$ as $\Theta(g)(2)(k)$ for $k<|\Theta(g)(2)|$, and the zero sequence otherwise.  
For the second part of the proof, we prove that 
$((\exists^{\st} \iota)\BET(\iota)\wedge \paai)\di \STP$ in $\P$.  Applying Theorem \ref{consresultcor} to this implication as in Theorem \ref{nogwel} yields the required term $s$.  
Now, for standard $\iota$ the formula $\BET(\iota)$ implies, since standard inputs yield standard outputs, that
\[\textstyle
(\forall^{\st} g^{2}, T^{1}\leq_{1}1)\big[(\forall \alpha \in \iota(g)(2))  (\overline{\alpha}g(\alpha)\not\in T)
\di  (\exists^{\st}k^{0})(\forall \alpha^{1}\leq_{1}1)(\exists i\leq k)(\overline{\alpha}i\not\in T)\big].
\]
Thanks to $\paai$, we may replace the antecedent by $(\forall^{\st} \alpha \in \zeta(g)(2))  (\overline{\alpha}g(\alpha)\not\in T)$, 
which is implied by $(\forall^{\st}\alpha\leq_{1}1)(\exists^{\st}n^{0})(\overline{\alpha}g(\alpha)\not\in T)$, and $\STP$ follows.  
The analogous results for $\chi$ as in $\ALP(\chi)$ follow from the observation that the consequent of the latter is equivalent to $ (\exists k^{0}) (\forall \alpha^{0^{*}}\leq_{0^{*}}1)(\exists i\leq k)(|\alpha|\leq k \di \overline{\alpha}i\not\in T)$, to which (the contraposition of) $\paai$ may be applied.  
\end{proof}
\begin{cor}
Any functional $\iota$ as in $\BET(\iota)$ \(resp.\ $\chi$ as in $\ALP(\chi)$\) is not computable in $(\exists^{2})$.
\end{cor}
Furthermore, the combination of $\iota$ and $\chi$ as above can compute the special fan functional, but we conjecture that neither functional \emph{alone} suffices.  
In this way, the special fan functional can easily be `split' in two similar but independent pieces.  Such a `splitting' is apparently difficult to obtain in Friedman-Simpson Reverse Mathematics (See e.g.\ \cite{splitting} for an example).  

\begin{rem}[Further ideas]\rm
Normann: We need far less than $(\E_{2})$ to capture the construction from the proof, but it may be difficult to isolate a simpler functional in which $\Theta$ is computable.

\medskip

Sam: What about the following one which is the `$\Delta_{1}^{1}$' version of $(\E_{2})$
\begin{align*}%\tag{$\mathcal{E}_{1,5}$}\label{hah2}
(\exists \zeta^{3})(\forall Y^{2}, Z^{2})\big[ (\forall m^{0})[(\exists & f^{1})(Y(f, m)=0) \asa (\forall g^{1})(Z(g,m)\ne0)] \\
& \di  (\forall n^{0})[(\exists f^{1})(Y(f,n)=0)\asa \zeta(Y, Z,n)=0]  \big].
\end{align*}

The above non-computability results are counterexamples to the heuristic
\begin{center}
\emph{First-order strength is roughly proportional to computational hardness}.  
\end{center}
present in the study of the computability of type one objects.

\medskip

Shall we call the behaviour of the (special) fan functional a `phase transition' (Andreas Weiermann coined this slogan for his work in incompleteness).  
\end{rem}

\subsection{A negative result in Nonstandard Analysis}
%As suggested in the previous section, the special fan functional originates from Nonstandard Analysis.
%In this section, we use the results from the previous section to prove a non-implication in Nonstandard Analysis.  
%
%\medskip
In section \ref{RM}, we observed that the Big Five of RM are linearly ordered as in \eqref{linord}.  
In this section, we show that the \emph{nonstandard} counterparts of $\FIVE$, $\ACA_{0}$, and $\WKL_{0}$ are however \emph{incomparable}.  
Surprisingly, we make essential use of Theorem \ref{import} to establish this result, rather than taking the `usual' model-theoretic route.  
Indeed, the fact that the full axiom \emph{Transfer} does not imply the full axiom \emph{Standard Part} is known (over various systems; see \cites{blaaskeswijsmaken, gordon2}), and is established using model-theoretic techniques.

\medskip

First of all, Nelson's system $\IST$ and the associated fragment $\P$ were introduced in Section \ref{base}.
The system $\P$ includes Nelson's axiom \emph{Idealisation} (formulated in the language of finite types), but to guarantee a conservative extension of Peano arithmetic, Nelson's axiom \emph{Transfer} must be omitted, while \emph{Standard Part} is weakened to $\HAC_{\INT}$.  Indeed, the fragment of \emph{Transfer} for $\Pi_{1}^{0}$-formulas as follows  
\be\tag{$\paai$}
(\forall^{\st}f^{1})\big[  (\forall^{\st}n)f(n)\ne0 \di (\forall m)f(m)\ne0  \big]
\ee
is the nonstandard counterpart of arithmetical comprehension (actually the equivalent $\Pi_{1}^{0}$-comprehension), while the fragment of \emph{Transfer} for $\Pi_{1}^{1}$-formulas as follows  
\be\tag{$\Paai$}
(\forall^{\st}f^{1})\big[ (\exists g^{1})(\forall x^{0})(f(\overline{g}n)=0)\di (\exists^{\st}g^{1})(\forall x^{0})(f(\overline{g}n)=0)\big]
\ee
is the nonstandard counterpart of $\FIVE$. 
The following fragment of \emph{Standard Part} is the nonstandard counterpart of weak K\"onig's lemma:
\be\tag{$\STP$}
(\forall \alpha^{1}\leq_{1}1)(\exists^{\st}\beta^{1}\leq_{1}1)(\alpha\approx_{1}\beta),
\ee  
where $\alpha\approx_{1}\beta$ is short for $(\forall^{\st}n)(\alpha(n)=_{0}\beta(n))$.  There is no deep philosophical meaning to be found in the words `nonstandard counterpart':  This is just what the principles $\STP$, $\paai$, and $\Paai$ are called in the literature (\cite{pimpson, sambon}).    

\medskip

Secondly, while $\FIVE\di \ACA_{0}\di \WKL_{0}$ by \eqref{linord},
we show in Theorem \ref{nogwel} and Corollary \ref{forqu} that the associated \emph{nonstandard implications} $\paai\di \STP$ and $\Paai\di \STP$ do not hold.  As noted above, we shall establish this non-implication using Theorem \ref{import}.  
To establish the aforementioned non-implications, we require the following theorem which provides a normal form for $\STP$ and establishes the latter's relationship with the special fan functional.  % is the functional version of $\STP$.
\begin{thm}\label{lapdog}
In $\P_{0}$, $\STP$ is equivalent to the following:
\begin{align}\label{frukkklk}
(\forall^{\st}g^{2})(\exists^{\st}w^{1^{*}})\big[(\forall T^{1}\leq_{1}1)(\exists ( \alpha^{1}\leq_{1}1,  &~k^{0}) \in w)\big((\overline{\alpha}g(\alpha)\not\in T)\\
&\di(\forall \beta\leq_{1}1)(\exists i\leq k)(\overline{\beta}i\not\in T) \big)\big]. \notag
\end{align}  
Furthermore, $\P_{0}$ proves $(\exists^{\st}\Theta)\SCF(\Theta)\di \STP$.
\end{thm}
\begin{proof}  
First of all, $\STP$ is easily seen to be equivalent to 
\begin{align}\label{fanns}
(\forall T^{1}\leq_{1}1)\big[(\forall^{\st}n)(\exists \beta^{0})&(|\beta|=n \wedge \beta\in T ) \di (\exists^{\st}\alpha^{1}\leq_{1}1)(\forall^{\st}n^{0})(\overline{\alpha}n\in T)   \big],
\end{align}
and this equivalence may also be found in \cite{samGH}*{Theorem 3.2}.  For completeness, we first prove the equivalence $\STP\asa \eqref{fanns}$.
Assume $\STP$ and apply overspill to $(\forall^{\st}n)(\exists \beta^{0})(|\beta|=n \wedge \beta\in T )$ to obtain $\beta_{0}^{0}\in T$ with nonstandard length $|\beta_{0}|$.  
Now apply $\STP$ to $\beta^{1}:=\beta_{0}*00\dots$ to obtain a \emph{standard} $\alpha^{1}\leq_{1}1$ such that $\alpha\approx_{1}\beta$ and hence $(\forall^{\st}n)(\overline{\alpha}n\in T)$.  
For the reverse direction, let $f^{1}$ be a binary sequence, and define a binary tree $T_{f}$ which contains all initial segments of $f$.  
Now apply \eqref{fanns} for $T=T_{f}$ to obtain $\STP$.    

\medskip

% and Theorem \ref{agda} below.
For the implication \eqref{frukkklk}$\di$\eqref{fanns}, note that \eqref{frukkklk} implies for all standard $g^{2}$
\begin{align}\label{frukkklk2}
(\forall T^{1}\leq_{1}1)(\exists^{\st} ( \alpha^{1}\leq_{1}1,  &~k^{0})\big[(\overline{\alpha}g(\alpha)\not\in T)
\di(\forall \beta\leq_{1}1)(\exists i\leq k)(\overline{\beta}i\not\in T) \big], 
\end{align}  
which in turn yields, by bringing all standard quantifiers inside again, that:
\begin{align}\label{frukkklk3}
(\forall T\leq_{1}1) \big[(\exists^{\st}g^{2})(\forall^{\st}\alpha \leq_{1}1)(\overline{\alpha}g(\alpha)\not\in T)\di(\exists^{\st}k)(\forall \beta\leq_{1}1)(\overline{\beta}k\not\in T) \big], 
\end{align}  
To obtain \eqref{fanns} from \eqref{frukkklk3}, apply $\HAC_{\INT}$ to $(\forall^{\st}\alpha^{1}\leq_{1}1)(\exists^{\st}n)(\overline{\alpha}n\not\in T)$ to obtain standard $\Psi^{1\di 0^{*}}$ such that  
$(\forall^{\st}\alpha^{1}\leq_{1}1)(\exists n\in \Psi(\alpha))(\overline{\alpha}n\not\in T)$, and defining $g(\alpha):=\max_{i<|\Psi|}\Psi(\alpha)(i)$ we obtain $g$ as in the antecedent of \eqref{frukkklk3}.  The previous implies 
\be\label{gundark}
(\forall T^{1}\leq_{1}1) \big[(\forall^{\st}\alpha^{1}\leq_{1}1)(\exists^{\st}n)(\overline{\alpha}n\not\in T)\di (\exists^{\st}k)(\forall \beta\leq_{1}1)(\overline{\beta}i\not\in T) \big], 
\ee
which is the contraposition of \eqref{fanns}, using classical logic.  For the implication $\eqref{fanns}  \di \eqref{frukkklk}$, consider the contraposition of \eqref{fanns}, i.e.\ \eqref{gundark}, and note that the latter implies \eqref{frukkklk3}.  Now push all standard quantifiers outside as follows:
\[
(\forall^{\st}g^{2})(\forall T^{1}\leq_{1}1)(\exists^{\st} ( \alpha^{1}\leq_{1}1, ~k^{0})\big[(\overline{\alpha}g(\alpha)\not\in T)
\di(\forall \beta\leq_{1}1)(\exists i\leq k)(\overline{\beta}i\not\in T) \big], 
\]
and applying idealisation \textsf{I} yields \eqref{frukkklk}.  The equivalence involving the latter also immediately establishes the second part of the theorem.    
\end{proof}
In light of the previous theorem, the `nonstandard' provenance of the special fan functional becomes clear.  This functional was actually discovered during the study of the Gandy-Hyland functional in Nonstandard Analysis in \cite{samGH}*{\S3-4}.  

\medskip

Thirdly, we establish the aforementioned non-implications and related results. 
\begin{thm}\label{nogwel}
The system $\P+\paai$ does not prove $\STP$.  
\end{thm}
\begin{proof}
Suppose $\P+\paai\vdash \STP$ and note that $\paai$ is equivalent to 
\be\label{trakke}
(\forall^{\st}f^{1})(\exists^{\st}n^{0})\big[ (\exists m)f(m)=0 \di (\exists i\leq n)f(i)=0  \big],
\ee
by contraposition.  Then the implication `$\paai\di \STP$' becomes
\be\label{jaj}
(\forall^{\st}f^{1})(\exists^{\st}n^{0})A(f,n)\di (\forall^{\st}g^{2})(\exists^{\st}w^{1^{*}})B(g,w)
\ee
where $B$ is the formula in square brackets in \eqref{frukkklk} and where $A$ is the formula in square brackets in \eqref{trakke}.  
We may strengthen the antecedent of \eqref{jaj} as follows:
\be\label{jaj2}
(\forall^{\st} h^{2})\big[(\forall^{\st}f^{1})A(f,h(f))\di (\forall^{\st}g^{2})(\exists^{\st}w^{1^{*}})B(g,w)\big], 
\ee  
In turn, we may strengthen the antecedent of \eqref{jaj2} as follows:
\be\label{jaj3}
(\forall^{\st} h^{2})\big[(\forall f^{1})A(f,h(f))\di (\forall^{\st}g^{2})(\exists^{\st}w^{1^{*}})B(g,w)\big], 
\ee  
Bringing out the standard quantifiers, we obtain
\be\label{jaj4}
(\forall^{\st} h^{2}, g^{2})(\exists^{\st}w^{1^{*}})\big[(\forall f^{1})A(f,h(f))\di B(g,w)\big], 
\ee  
and applying Corollary \ref{consresultcor} to `$\P\vdash \eqref{jaj4}$', we obtain a term $t$ such that         
\be\label{jaj5}
(\forall h^{2}, g^{2})(\exists w^{1^{*}}\in t(h,g))\big[(\forall f^{1})A(f,h(f))\di B(g,w)\big], 
\ee  
is provable in $\textsf{E-PA}^{\omega*}$.  Clearly, the antecedent of \eqref{jaj5} expresses that $h$ is Feferman's search functional $(\mu^{2})$.  
Furthermore, it is straightforward to define $\Theta$ as in $\SCF(\Theta)$ in terms of $t(h,\cdot)$;  However, this implies that the special fan functional is computable in $(\mu^{2})$ via a term from G\"odel's $\textsf{T}$.  This contradicts Corollary \ref{import2}, and we are done.   
\end{proof}
\begin{cor}\label{nogwel2}
The system $\P+\WKL^{\st}$ does not prove $\STP$.  The same holds for $\P+\varphi+\paai$, where $\varphi$ is any internal sentence such that the latter system is consistent.  
\end{cor}
\begin{proof}
For the first part, note that $\paai\di (\mu^{2})^{\st}\di \WKL^{\st}$ where the second implication follows from the usual proof of $\ACA_{0}\di \WKL_{0}$ relative to `st', and where the first implication 
follows from applying $\HAC_{\INT}$ to \eqref{trakke} (and taking the maximum of all outputs of the resulting functional).  
  
\medskip  
  
For the second part, suppose $\P+\varphi+\paai\vdash \STP$ and apply Theorem \ref{consresultcor} to $\P+\varphi+\paai\vdash \eqref{frukkklk}$ to obtain a term of G\"odel's \textsf{T} which computes the special fan functional in terms of the Turing jump functional.  This contradiction immediately yields the second part of the theorem.      
\end{proof}
%Note that one can replace $\WKL$ in the second part by \emph{any true internal sentence}.  
\begin{cor}\label{forqu}
The system $\P+\Paai$ does not prove $\STP$.  
\end{cor}
\begin{proof}
Follows from Corolllary \ref{import2} in the same way as the theorem.  By way of a sketch, suppose that $P\vdash \Paai\di \STP$.  Then  
\end{proof}
In the same way, Corollary \ref{import2} yields that \emph{Transfer} limited to $\Pi_{k}^{1}$-formulas cannot imply $\STP$.  
Indeed, the `comprehension functional' for $\Pi_{1}^{k}$-formulas has type two, and hence does not compute the special fan functional by Corollary \ref{import2}. 

\medskip

Despite the above negative results, we have the following conservation result.  % where $\RCA_{0}^{2}$ is just $\RCA_{0}$ formulated in the language of finite types (See \cite{kohlenbach2}*{\S2} for the exact definition).  
\begin{thm}\label{conske}
The systems $\RCA^{\omega}_{0}+(\exists \Theta)\SCF(\Theta)$ and $\P_{0}+\STP$ are conservative over $\RCA_{0}^{2}+\WKL$ for sentences in the latter's second-order language.  
\end{thm}    
\begin{proof}
For the first system, by \cite{samGH}*{Cor.\ 3.4} and Theorem \ref{kinkel}, the special fan functional can be defined in terms of $\Phi$ as in $\MUC(\Phi)$.  
By Theorem \ref{protofinal}, the first conservation result is now immediate.  For the second conservation result, let $\varphi$ be a (necessarily internal) sentence in the language of $\RCA_{0}^{2}+\WKL$.  If $\P_{0}+\STP\vdash \varphi$, then $\P_{0}\vdash (\exists^{\st}\Theta)\SCF(\Theta)\di \varphi$ by the second part of Theorem \ref{lapdog}.  Applying Corollary \ref{consresultcor} to $\P_{0}\vdash (\forall^{\st}\Theta)(\SCF(\Theta)\di \varphi)$ yields $\RCA_{0}^{\omega}\vdash (\forall \Theta)(\SCF(\Theta)\di \varphi) $.  % and we are done.
\end{proof}
Finally, we also establish the `reverse' direction of the fact that Theorem \ref{import} gives rise to Theorem \ref{nogwel}.  
\begin{thm}
Suppose $\P+\varphi\not\vdash \paai\di \STP$ for some internal sentence $\varphi$.  Then for every term $t$ from G\"odel's \textsf{\textup{T}},  $\textsf{\textup{E-PA}}^{\omega*}+\varphi$ does not prove that $t$ expresses a special fan functional in terms of the special fan functional.
\end{thm}
\begin{proof}
X
\end{proof}

\subsection{Compactness}



Anti-specker from $(\forall x\in [0,1])(\exists^{\st}y)(x\approx y)$


\medskip

NSA and INT share notion of compactness!  

\medskip


\medskip\newpage

One of the various definitions of compactness states that `If a point $z$ is bounded away from a compact space $X$, $z$ is uniformly bounded away from $X$'.  
We consider the following version of this kind of compactness where $\kappa^{2\di (0\times 1^{*})}$ provides the uniform bound (namely $\kappa(g)(1)$) for $z$ and the unit interval, and a finite sequence of reals in the unit interval (namely $\kappa(g)(2)=(y_{0}, y_{1}, \dots, y_{k})$) which have to be bounded away from $z$ as given by $g$.   
%for which have to be bounded away from $z$.  
\begin{align}\textstyle
(\forall g^{2}, z\in \R)
\Big[\big[ \big(\forall y\in ( \kappa(g)(2)\cap [0,1])\big)&\textstyle (|y-z|>_{\R} \frac{1}{g(y)})\tag{$\textsf{COMP}([0,1], \kappa)$} \label{kalkuttttt2}\\
&\textstyle\di (\forall x\in [0,1])(|x-z|>_{\R}\frac{1}{\kappa(g)(1)})   \big]\Big],\notag
\end{align}


%In light of the previous theorem, the existence of the special fan functional is not really stronger than $\WKL_{0}$.  
%In conclusion, we have introduced the \emph{special fan functional}, inspired by {\STP}, and \eqref{frukkklk} in particular.  
%This functional turned out to be not computable in $(\exists^{2})$, leading to the non-implication $\paai\not\di\STP$, which is a part of the folklore of Nonstandard Analysis.  
%
\subsection{A special case of the special fan functional}
In this section, we consider the principle \emph{weak weak K\"onig's lemma}, $\WWKL$ for short, first introduced in \cite{yussie} and defined as in Definition \ref{leipi}.
We shall study the nonstandard counterpart of $\WWKL$ as in Definition \ref{leipi2}, and the associated weak version of the special fan functional.  

\medskip

First of all, we have the following definitions.  
\bdefi[Weak weak K\"onig's lemma]\label{leipi}~
\begin{enumerate}
\item For a binary tree, define $\mu(T):=\lim_{n\di \infty}\frac{\{\sigma \in T: |\sigma|=n    \}}{2^{n}}$.
\item For a binary tree $T$, define `$\mu(T)>_{\R}a^{1}$' as $(\exists k^{0})(\forall n^{0})\big(\frac{\{\sigma \in T: |\sigma|=n    \}}{2^{n}}\geq a+\frac{1}{k}\big)$.
\item We define $\WWKL$ as 
%\be\tag{$\WWKL$}\label{WWKL}
\[
(\forall T \leq_{1}1)\big[ \mu(T)>_{\R}0\di (\exists \beta\leq_{1}1)(\forall m)(\overline{\beta}m\in T) \big],
\]
%\ee
\item Define $\WFAN$ as the classical contraposition of $\WWKL$.  
\end{enumerate}     
\edefi
%Note that the formula `$\mu(T)>_{\R} a$' makes sense even when the limit $\mu(T)$ does not exist.  
Although $\WWKL$ is not part of the `Big Five' systems of RM, there are \emph{some} equivalences involving the former (\cite{yussie, sayo, yuppie, simpson2}).    
The nonstandard counterpart of $\WWKL$ was first introduced in \cite{kei1} as follows:  
\bdefi[Nonstandard $\WWKL$]\label{leipi2}~
Let $T$ be a binary tree.  
\begin{enumerate}
%\item For a binary tree, define $\mu(T):=\lim_{n\di \infty}\frac{\{\sigma \in T: |\sigma|=n    \}}{2^{n}}$.
\item Define `$\mu(T)\gg a^{1}$' as $(\exists^{\st} k^{0})(\forall^{\st} n^{0})\big(\frac{\{\sigma \in T: |\sigma|=n    \}}{2^{n}}\geq a+\frac{1}{k}\big)$.
\item Define `$\mu(T)\approx 0$' as $(\forall^{\st} k^{0})(\exists^{\st} n^{0})\big(\frac{\{\sigma \in T: |\sigma|=n    \}}{2^{n}}< \frac{1}{k}\big)$.
\item Define the `Loeb measure of $T$' as $L_{N}(T):=\frac{\{\sigma \in T: |\sigma|=N   \}}{2^{N}}$.
\item Define $\LMP$ as 
%\be\tag{$\WWKL$}\label{WWKL}
\be\label{fanzzz}
(\forall   T \leq_{1}1)\big[  (\exists N\in \Omega)(L_{N}(T)\gg 0) \di (\exists^{\st} \beta\leq_{1}1)(\forall^{\st}n)(\overline{\beta}n\in T)\big].
%(forall T \leq_{1}1)\big[ \mu(T)>_{\R}0\di (\exists \beta\leq_{1}1)(\forall m)(\overline{\beta}m\in T) \big],
\ee

\end{enumerate}     
\edefi
Clearly, $\WWKL$ and $\LMP$ are weakenings of $\WKL$ and $\STP$.  Similarly, we introduce the following weak version of the special fan functional. 
\bdefi[Weak special fan functional] 
We define $\WCF(\Lambda)$ for $\Lambda^{(2\di (1\times 1^{*}))}$:
\[\textstyle
(\forall k^{0},g^{2}, T^{1}\leq_{1}1)\big[(\forall \alpha \in \Lambda(g,k)(2))  (\overline{\alpha}g(\alpha)\not\in T)
\di   (\exists n\leq\Lambda(g,k)(1) ) (L_{n}(T)\leq\frac{1}{k})\big].
\]
%where `$T^{1}\leq_{1}1$' is meant to convey that $T$ is a binary tree, and where $1^{*}$ is the type of finite sequences of type $1$. 
Any functional $\Lambda$ satisfying $\WCF(\Lambda)$ is referred to as a \emph{weak special fan functional}.
\edefi

We first obtain a normal form for $\LMP$ as follows.  
\begin{thm}\label{lapdoc}
In $\P_{0}$, the principle $\LMP$ is equivalent to:
\begin{align}\label{w2}\textstyle
(\forall^{\st}g^{2},k^{0})&(\exists^{\st}w^{1^{*}})\\
&\textstyle\notag\big[(\forall T\leq_{1}1)(\exists ( \alpha\leq_{1}1, n) \in w)\big(\overline{\alpha}g(\alpha)\not\in T\di L_{n}(T)\leq\frac1k\big)\big].
%&\di(\forall \beta\leq_{1}1)(\exists i\leq k)(\overline{\beta}i\not\in T) \big)\big]. \notag
\end{align}  
Furthermore, $\P_{0}$ proves $(\exists^{\st} \Lambda)\WCF(\Lambda)\di \LMP$.  
%In $\RCAO$, $\STP\di \LMP\di \WWKL^{\st}$ and $\LMP\di \WWKL_{\ns}$.  
\end{thm}
\begin{proof}
Analogous to the proof of Theorem \ref{lapdog}. 
\end{proof}
We have the following expected theorem.
\begin{thm}\label{import22}
Any functional $\Lambda^{3}$ as in $\WCF(\Lambda)$ is not computable in $(\exists^{2})$ \(or any type two functional\).   
\end{thm}
\begin{proof}
Analogous to the proof of Theorem \ref{import}.  In fact, the only required modification is that tree $T_{0}$ in the proof of the latter just needs to satisfy $\mu(T)>_{\R}0$.  
\end{proof}
 \begin{cor}
The system $\P+\paai+\WWKL^{\st}$ does not prove $\LMP$.  
\end{cor}
\begin{proof}
Similar to Theorem \ref{nogwel}; note that $\paai\di (\exists^{2})^{\st}\di \WWKL^{\st}$.
\end{proof}
As to the logical strength of $\WCF(\Lambda)$, we will establish that the latter yields a conservative extension of $\WWKL$, similar to Theorems \ref{proto} and \ref{kinkel}.  
To this end, consider the following uniform and nonstandard principles.  
\be\tag{$\PUC(\kappa)$}\textstyle
(\forall Y^{2},k^{0})(\forall n\geq \kappa(Y)(k))\big( \frac{| \{ \tau\leq_{0^{*}}1 : |\tau|=n\wedge Y(\hat{\tau})> n  \}   |}{2^{n}}   \leq\frac{1}{k}   \big),   % 
%(\forall Y^{2},k^{0})(\forall n\geq \kappa(Y)(k))\big( \frac{| \{ \tau\leq_{0^{*}}1 : (\exists \rho\leq_{0^{*}} 1)(|\rho|,|\tau|=n\wedge Y(\hat{\tau})\ne Y(\hat{\rho}) )  \}   |}{2^{n}}   \leq\frac{1}{2^{k}}   \big),   % 
%\label{lukl3}\tag{$\textsf{\textup{MUC}}(\Phi)$}$
\ee
\be\tag{$\PUC_{\ns}$}\textstyle
(\forall^{\st} Y^{2})(\forall N\in \Omega)\big( \frac{| \{ \tau\leq_{0^{*}}1 : |\tau|=N\wedge Y(\hat{\tau})>N \}   |}{2^{N}}  \approx 0  \big),   % \label{lukl3}\tag{$\textsf{\textup{MUC}}(\Phi)$}$
\ee
where $\hat{\sigma}:=\sigma*00\dots$ for a finite sequence $\sigma$.  
Intuitively speaking, $\PUC_{\ns}$ expresses that the probability that $Y$ is nonstandard at some sequence is infinitesimal.  
We refer to $\kappa$ as in $\PUC(\kappa)$ as the \emph{weak intuitionistic fan functional}.  

\medskip
   
We have the following results similar to Theorems \ref{proto} and \ref{kinkel} . % where all systems are introduced in \cite{kohlenbach2}*{\S2}.  
\begin{thm}\label{proto2}
The system $\RCA_{0}^{\omega}+(\exists \kappa^{3})\PUC(\kappa)$ is a conservative extension of $\RCA_{0}^{2}+\WWKL$ \(for the latter's second-order language\).   
\end{thm}
\begin{proof}
A proof may be found in Theorem \ref{photon} of the Appendix. 
\end{proof}
\begin{thm}
The system $\P_{0}$ proves $\PUC_{\ns}\di \LMP$.   From this proof, a term $t$ can be extracted such that $\RCAo$ proves $(\forall \kappa)\big(\PUC(\kappa)\di \WCF(t(\kappa))\big)$.
\end{thm}
\begin{proof}
For the first part, consider the contraposition of $\LMP$ as follows:
\[
(\forall   T \leq_{1}1)\big[  (\forall^{\st} \beta\leq_{1}1)(\exists^{\st}n)(\overline{\beta}n\not\in T)\di  (\forall N\in \Omega)(L_{N}(T)\approx 0) \big].
\]
If the antecedent $(\forall^{\st} \beta\leq_{1}1)(\exists^{\st}n)(\overline{\beta}n\not\in T)$ holds, apply $\HAC_{\INT}$ to obtain standard $Y^{2}$ such that 
$(\forall^{\st} \beta\leq_{1}1)(\exists n\leq Y(\beta) )(\overline{\beta}n\not\in T)$.  By $\PUC_{\ns}$, we have that $ \frac{1}{2^{N}}{| \{ \tau\leq_{0^{*}}1 : |\tau|=N\wedge Y(\hat{\tau})>N \}   |}  \approx 0  $ for nonstandard $N$.  By definition, we also have the following for nonstandard $N$:
\[\textstyle
1\approx  \frac{| \{ \tau\leq_{0^{*}}1 : |\tau|=N\wedge Y(\hat{\tau})\leq N \}   |}{2^{N}}  \leq   \frac{| \{ \tau\leq_{0^{*}}1 : |\tau|=N\wedge  \tau \not  \in T \}   |}{2^{N}}=1-L_{N}(T), 
\]
which yields that $L_{N}(T)\approx 0$, and $\LMP$ follows.  For the second part of the proof, note that $\LMP$ has an equivalent normal form \eqref{w2}, while an (equivalent) normal form for $\PUC_{\ns}$ is as follows:
\be\textstyle\label{dz}
(\forall^{\st} Y^{2}, k^{0})(\exists^{\st}M^{0})(\forall N\geq M)\big( \frac{| \{ \tau\leq_{0^{*}}1 : |\tau|=N\wedge Y(\hat{\tau})>N \}   |}{2^{N}}  \leq \frac{1}{{k}}  \big),   % \label{lukl3}\tag{$\textsf{\textup{MUC}}(\Phi)$}$
\ee
which one easily obtains using underspill.  Now proceed as in the proof of Theorem~\ref{nogwel} to obtain the relative computability result from the theorem.  In particular, bring $\eqref{w2}\di \eqref{dz}$ into normal form as in the aforementioned proof and apply Theorem \ref{consresultcor} to obtain the desired term.  
\end{proof}
The nonstandard proof in the theorem is rather trivial.  We have the following immediate corollary.  
\begin{cor}\label{proto3}
The systems $\RCA_{0}^{\omega}+(\exists \Lambda^{3})\WCF(\Lambda)$ and $\P_{0}+\LMP$ are conservative extensions of $\RCA_{0}^{2}+\WWKL$ \(for the latter's second-order language\).  
\end{cor}

The following corollary establishes the nonstandard version of the non-implication $\WWKL\not\di \WKL$, which was first proved in \cite{yussie}.  
 \begin{cor}
%For a given term $t$ of G\"odel's $\textsf{\textup{T}}$, $\textsf{\textup{E-PRA}}^{\omega*}$ does not prove $(\forall \Lambda)(\WCF(\Lambda)\di \SCF(t(\Lambda)))$.  
The system $\P_{0}+\LMP$ does not prove $\STP$.  
\end{cor}
\begin{proof}
We proceed similar to Theorem \ref{nogwel}.   Suppose $\P+\LMP\vdash \STP$;  in the same way as for the aforementioned theorem,
we obtain some term $t$ such that $\RCAo$ proves $(\forall \Lambda)(\WCF(\Lambda)\di \SCF(t(\Lambda)))$.  % in $\textsf{E-PA}^{\omega*}$ for some term $t$.   
In particular $\RCAo+(\exists \Lambda)\WCF(\Lambda)$ proves $(\exists \Theta)\SCF(\Theta)$.
Since $(\exists \Theta)\SCF(\Theta)\di \WKL$ over $\RCAo$, we have that $\RCAo+(\exists \Lambda)\WCF(\Lambda)$ proves $\WKL$, contradicting Corollary \ref{proto3}.
\end{proof}
The following corollary is now straightforward.
 \begin{cor}
For any term $t$ of G\"odel's $\textsf{\textup{T}}$, $\textsf{\textup{E-PA}}^{\omega*}\not\vdash(\forall \Lambda)(\WCF(\Lambda)\di \SCF(t(\Lambda)))$.  
\end{cor}
The following theorem generalises the previous result.
\begin{thm}\label{import24}
Any $\Theta^{3}$ as in $\SCF(\Theta)$ is not computable in $\Lambda$ such that $\WCF(\Lambda)$.   
\end{thm}
\begin{proof}
?
\end{proof}

Finally, we discuss a version of the weak special fan functional more similar to $\WWKL$.  To bring out the similarity to the latter, we use write `$(\forall \alpha^{1} \in \gamma)\varphi(\alpha)$' instead of `$(\forall n^{0})\varphi(\gamma(n))$' for $\gamma^{0\di 1}$. 
\bdefi[Weak weak special fan functional] 
Define $\WWF(\zeta)$ for $\zeta^{(2\di (1\times (0\di 1)))}$:
\[\textstyle
(\forall g^{2}, T^{1}\leq_{1}1)\big[(\forall \alpha \in \zeta(g)(2))  (\overline{\alpha}g(\alpha)\not\in T)
\di   (\forall k^{0})(\exists n\leq\zeta(g)(1)(k) ) (L_{n}(T)\leq\frac{1}{k})\big].
\]
%where `$T^{1}\leq_{1}1$' is meant to convey that $T$ is a binary tree, and where $1^{*}$ is the type of finite sequences of type $1$. 
Any functional $\zeta$ as in $\WWF(\zeta)$ is referred to as a \emph{weak weak special fan functional}.
\edefi
\begin{thm}
There are terms $s, t$ such that $\textsf{\textup{E-PA}}^{\omega*}$ proves $(\forall \Lambda)(\WCF(\Lambda)\di \WWF(t(\Lambda)))$ 
and $(\forall \zeta, \mu)\big((\WWF(\zeta)\wedge \MU(\mu))\di \WCF(s(\zeta,\mu))\big)$. 
\end{thm}
\begin{proof}
For the first part of the proof, define $t(\Lambda)(g)(1):=(\lambda k)\Lambda(g,k)(1)$ and $t(\Lambda)(g)(2):= (\lambda k)\Lambda(g,k)(2)$.  For the second part of the proof, we prove that 
$((\exists^{\st} \zeta)\WWF(\zeta)\wedge \paai)\di \LMP$ in $\P$.  Applying Theorem \ref{consresultcor} to this implication as in Theorem \ref{nogwel} yields the required term $s$.  
Now, for standard $\zeta$ the formula $\WWF(\zeta)$ implies, since standard inputs yield standard outputs, that
\[\textstyle
(\forall^{\st} g^{2})(\forall T^{1}\leq_{1}1)\big[(\forall \alpha \in \zeta(g)(2))  (\overline{\alpha}g(\alpha)\not\in T)
\di   (\forall^{\st} k^{0})(\exists^{\st} n ) (L_{n}(T)\leq\frac{1}{k})\big].
\]
Thanks to $\paai$, we may replace the antecedent by $(\forall^{\st} \alpha \in \zeta(g)(2))  (\overline{\alpha}g(\alpha)\not\in T)$, 
which is implied by $(\forall^{\st}\alpha\leq_{1}1)(\exists^{\st}n^{0})(\overline{\alpha}g(\alpha)\not\in T)$, and $\LMP$ follows.   
\end{proof}

TEST

\appendix

\section{Conservation results}\label{conssec}

In this section, we provide proof for the conservation results in Theorems \ref{proto} and~\ref{proto2}.  
To this end, we require some definitions, starting with the notion of \emph{associate} (called `code' in Reverse Mathematics; see \cite{kohlenbach4}*{\S4}).  
\bnota
It is customary to define $\alpha(\sigma)$ for $\alpha^{1}$ and $\sigma^{0^{*}}$ by $\alpha(\pi(\sigma))$ where $\pi^{0^{*}\di 0}$ is some fixed function coding finite sequences into numbers.  Similarly, we define $\tau(\sigma)$ for $\tau^{0^{*}},\sigma^{0^{*}}$ as $1+\tau(\pi(\sigma))$ if $\pi(\sigma)<|\sigma|$, and zero otherwise.  
\enota
\bdefi[Associate of a continuous functional]
The sequence $\alpha^{1}$ is an \emph{associate} for the continuous type two functional $Y^{2}$ if
\begin{enumerate} 
\item $(\forall \beta^{1})(\exists n)\big(\alpha(\overline{\beta}n)>0\big)$,
\item $(\forall \beta^{1}, m^{0})(  \alpha(\overline{\beta}m)>0\di Y(\beta)+1=\alpha(\overline{\beta}n))$.
\end{enumerate}
\edefi
We also define the notion of associate independently, i.e.\ without referring to the functional it is representing.  
\bdefi[Associate]\label{wass}
The sequence $\alpha^{1}$ is an \emph{associate} if 
\begin{enumerate} 
\item $(\forall \beta^{1})(\exists n)\big(\alpha(\overline{\beta}n)>0\big)$,
\item $(\forall \sigma_{0}^{0^{*}}, \sigma_{1}^{0^{*}})(\alpha({\sigma_{0}})>0 \wedge \sigma_{0}\preceq \sigma_{1} \di \alpha({\sigma_{0}})=_{0}\alpha(\sigma_{1}) )   $.
\end{enumerate}
\edefi
Note that $\sigma\preceq\tau$ if $|\sigma|\leq |\tau|\wedge (\forall n<|\sigma|)(\sigma(n)=\tau(n))$, i.e.\ $\sigma$ is an initial segment of $\tau$.  
The second condition is also referred to as $\alpha$ being a `neighbourhood function'.  

\begin{thm}\label{protofinal}
The system $\RCA_{0}^{\omega}+(\exists \Omega^{3})\MUC(\Omega)$ is a conservative extension of $\RCA_{0}^{2}+\WKL$.  
\end{thm}
\begin{proof}
The theorem is listed in \cite{kohlenbach2}*{Prop.\ 3.15} and Kohlenbach states that for its proof, one can adapt the proof of \cite{troelstra1}*{Theorem 2.6.6, p.\ 141}.  
We discuss the latt`er proo'f and its modification in more detail.  %We focus on intuitive understanding rather than technical detail.    

\medskip

Now, \cite{troelstra1}*{Theorem 2.6.6} essentially states that every model $\mathcal{U}$ of the `full' fan theorem can be extended to a model $\textsf{ECF}(\mathcal{U})$ which includes a fan functional as in $\MUC(\Omega)$.  % where $\Omega(Y)$ is the least point of uniform continuity.  
The `full' fan theorem is an intuitionistic principle defined as follows (See \cite{troelstra1}*{1.9.24}):  For every formula $A$, we have that
\[
(\forall \alpha\leq_{1}1)(\exists x^{0})A(\alpha, x)\di (\exists z^{0})(\forall \alpha^{1}\leq_{1}1)(\exists y^{0} )(\forall\beta\leq_{1}1)(\overline{\alpha}z=_{0}\overline{\beta}z\di A(\beta, y)).  
\]
As an aside, the `full' fan theorem implies\footnote{Take $A(\alpha, n)$ to be $\overline{\alpha}n\not\in T$ for a binary tree $T$, and $\FAN$ follows.} $\FAN$, i.e.\ the classical contraposition of $\WKL$, but contradicts classical mathematics\footnote{Take $A(\alpha, n)$ to be $(\forall \gamma\leq_{1}1)(\overline{\alpha}n=\overline{\beta}n\di Y(\alpha)=Y(\beta))$ for any $Y^{2}$, and note that all type two functionals are thus continuous on Cantor space.}.  

\medskip

From the proofs of \cite{troelstra1}*{Theorems~2.6.4 and~2.6.6}, it is clear that the initial model $\mathcal{U}$ only needs to satisfy $\FAN$, the classical contraposition of $\WKL$, and \emph{not the `full' fan theorem}.  
In paticular, any model $\mathcal{U}$ satisfying $\FAN$ can be extended to a model satisfying the existence of a fan functional;  This sketch is the conceptual core of the proof of Theorem \ref{proto}.  

\medskip

Secondly, we discuss \emph{how} the fan functional is represented in $\textsf{ECF}(\mathcal{U})$.  For an associate $\alpha^{1}$, $(\forall \beta\leq_{1}1)(\exists n^{0})(\alpha(\overline{\beta}n)>0)$ implies $(\exists k^{0})(\forall \beta\leq_{1}1)(\exists n^{0}\leq k)(\alpha(\overline{\beta}n)>0)$ by $\FAN$, and hence a fan functional $\varphi_{\textup{uc}}$ can be defined as:
\[
\varphi_{\textup{uc}}(\alpha):=(\mu k^{0})(\forall \beta\leq_{1}1)(\exists n^{0}\leq k)(\alpha(\overline{\beta}n)>0).
\]
It remains to be shown that $\varphi_{\textup{uc}}$ is represented by an object in $\textsf{ECF}(\mathcal{U})$.  
This representation is called $[\varphi_{\textup{uc}}]'$ in the proof of \cite{troelstra1}*{Theorem 2.6.4} and defined as:  
%
\[
[\varphi_{\textup{uc}}]'(\sigma^{0^{*}}):=
(\mu m\leq z)(\forall \tau^{0^{*}}, \rho^{0^{*}}\leq_{0^{*}}1)\big((|\tau|=|\rho|=z ~\wedge~ \overline{\tau}m=\overline{\rho}m)\di \sigma(\tau)=\sigma(\rho)>0 \big)
\]
in case $(\exists k \leq |\sigma|)(\forall \theta^{0^{*}}\leq_{0^{*}}1)(|\theta|=k\di \sigma(\theta)>0)$ and $z$ is the least such number; otherwise, the number $[\varphi_{\textup{uc}}]'(\sigma^{0^{*}})$ is defined as zero.  Finally, it is straightforward to verify that $[\varphi_{\textup{uc}}]'$ is extensional in the sense required by $\textsf{ECF}(\mathcal{U})$.  % and we are done.    
%
%\medskip
%
%Note that if $\alpha^{1}$ is an associate and $[\varphi_{\textup{uc}}]'(\overline{\alpha}N)>0$, then $N$ is large enough to make sure $\alpha(\t)$
\end{proof}
Based on the previous proof, we now obtain the following theorem.
\begin{thm}\label{photon}
The system $\RCA_{0}^{\omega}+(\exists \kappa^{3})\PUC(\kappa)$ is a conservative extension of $\RCA_{0}^{2}+\WWKL$.  
\end{thm}
\begin{proof}
For an associate $\alpha^{1}$, we have $(\forall \beta\leq_{1}1)(\exists n^{0})(n\geq\alpha(\overline{\beta}n)>0)$, where the inequality `$\geq$' follows from its status as a neighbourhood function.   
Now define a binary tree $T$ by $\sigma\in T_{0}\asa [\alpha(\sigma)=0\vee \alpha(\sigma)>|\sigma|]$.  
By $\WFAN$, we have $(\forall k^{0})(\exists n^{0})\big(\frac{|\{\sigma\in T_{0}:|\sigma|=n\}|}{2^{n}}\leq\frac{1}{k}\big)$, and hence we define the functional $\varphi_{\textup{puc}}$ as
\[\textstyle
\varphi_{\textup{puc}}(\alpha, k):=(\mu n^{0})\big(\frac{|\{\sigma:|\sigma|=n\wedge [\alpha(\sigma)=0\vee \alpha(\sigma)>|\sigma|]\}|}{2^{n}}\leq\frac{1}{k}\big),
\]
which is as required for the weak intuitionistic fan functional.  
It remains to be shown that $\varphi_{\textup{puc}}$ is represented by an object in $\textsf{ECF}(\mathcal{U})$.  
We shall denote this representation by $[\varphi_{\textup{puc}}]$, following the proof of Theorem \ref{protofinal}.    
\[\textstyle
[\varphi_{\textup{puc}}](\tau, k):=(\mu n^{0}\leq |\tau|)\big(\frac{|\{\sigma:|\sigma|=n\wedge [\alpha(\sigma)=0\vee \alpha(\sigma)>|\sigma|]\}|}{2^{n}}\leq\frac{1}{k}\big).
\]
if such $n$ exists, and zero otherwise.  
\end{proof}
\begin{ack}\rm
This research was supported by the following funding bodies: FWO Flanders, the John Templeton Foundation, the Alexander von Humboldt Foundation, the University of Oslo, and the Japan Society for the Promotion of Science.  
The authors express his gratitude towards these institutions. 
%The author would like to thank Grigori Mints, Ulrich Kohlenbach, Horst Osswald, Helmut Schwichtenberg, Stephan Hartmann, and Dag Normann for their valuable advice.  
%JTF, JSPS, Humboldt, FWO
% Kohlenbach, Fujiwara, Yamazaki, Benno van den Berg
\end{ack}





\begin{bibdiv}
\begin{biblist}
%\bibselect{allkeida}
\bib{avi2}{article}{
  author={Avigad, Jeremy},
  author={Feferman, Solomon},
  title={G\"odel's functional \(``Dialectica''\) interpretation},
  conference={ title={Handbook of proof theory}, },
  book={ series={Stud. Logic Found. Math.}, volume={137}, },
  date={1998},
  pages={337--405},
}

\bib{avi1}{article}{
  author={Avigad, Jeremy},
  author={Helzner, Jeremy},
  title={Transfer principles in nonstandard intuitionistic arithmetic},
  year={2002},
  journal={Archive for Mathmatical Logic},
  volume={41},
  pages={581--602},
}

\bib{brie}{article}{
  author={van den Berg, Benno},
  author={Briseid, Eyvind},
  author={Safarik, Pavol},
  title={A functional interpretation for nonstandard arithmetic},
  journal={Ann. Pure Appl. Logic},
  volume={163},
  date={2012},
  number={12},
  pages={1962--1994},
}

\bib{blaaskeswijsmaken}{article}{
  author={Blass, Andreas},
  title={End extensions, conservative extensions, and the Rudin-Frol\'\i k ordering},
  journal={Trans. Amer. Math. Soc.},
  volume={225},
  date={1977},
  pages={325--340},
}

\bib{damirzoo}{misc}{
  author={Dzhafarov, Damir D.},
  title={Reverse Mathematics Zoo},
  note={\url {http://rmzoo.uconn.edu/}},
}

\bib{splitting}{article}{
   author={Dzhafarov, Damir D.},
   author={Hirst, Jeffry L.},
   author={Lakins, Tamara J.},
   title={Ramsey's theorem for trees: the polarized tree theorem and notions of stability},
   journal={Arch. Math. Logic},
   volume={49},
   date={2010},
   number={3},
   pages={399--415},
%   issn={0933-5846},
%   review={\MR{2609990 (2011h:03018)}},
%   doi={10.1007/s00153-010-0179-6},
}

\bib{fega}{article}{
  author={Ferreira, Fernando},
  author={Gaspar, Jaime},
  title={Nonstandardness and the bounded functional interpretation},
  journal={Ann. Pure Appl. Logic},
  volume={166},
  date={2015},
  number={6},
  pages={701--712},
}

\bib{fried}{article}{
  author={Friedman, Harvey},
  title={Some systems of second order arithmetic and their use},
  conference={ title={Proceedings of the International Congress of Mathematicians (Vancouver, B.\ C., 1974), Vol.\ 1}, },
  book={ },
  date={1975},
  pages={235--242},
}

\bib{fried2}{article}{
  author={Friedman, Harvey},
  title={ Systems of second order arithmetic with restricted induction, I \& II (Abstracts) },
  journal={Journal of Symbolic Logic},
  volume={41},
  date={1976},
  pages={557--559},
}

\bib{gordon2}{article}{
  author={Gordon, Evgeni\v {\i } I.},
  title={Relatively standard elements in Nelson's internal set theory},
  journal={Siberian Mathematical Journal},
  volume={30},
  number={1},
  pages={68--73},
}

\bib{ishi1}{article}{
  author={Ishihara, Hajime},
  title={Reverse mathematics in Bishop's constructive mathematics},
  year={2006},
  journal={Philosophia Scientiae (Cahier Sp\'ecial)},
  volume={6},
  pages={43-59},
}

\bib{kohlenbach3}{book}{
  author={Kohlenbach, Ulrich},
  title={Applied proof theory: proof interpretations and their use in mathematics},
  series={Springer Monographs in Mathematics},
  publisher={Springer-Verlag},
  place={Berlin},
  date={2008},
  pages={xx+532},
}

\bib{kohlenbach2}{article}{
  author={Kohlenbach, Ulrich},
  title={Higher order reverse mathematics},
  conference={ title={Reverse mathematics 2001}, },
  book={ series={Lect. Notes Log.}, volume={21}, publisher={ASL}, },
  date={2005},
  pages={281--295},
}

\bib{kohlenbach4}{article}{
  author={Kohlenbach, Ulrich},
  title={Foundational and mathematical uses of higher types},
  conference={ title={Reflections on the foundations of mathematics (Stanford, CA, 1998)}, },
  book={ series={Lect. Notes Log.}, volume={15}, publisher={ASL}, },
  date={2002},
  pages={92--116},
}

\bib{kooltje}{article}{
  author={Kohlenbach, Ulrich},
  title={On uniform weak K\"onig's lemma},
  note={Commemorative Symposium Dedicated to Anne S. Troelstra (Noordwijkerhout, 1999)},
  journal={Ann. Pure Appl. Logic},
  volume={114},
  date={2002},
  number={1-3},
  pages={103--116},
}

\bib{longmann}{book}{
  author={Longley, John},
  author={Normann, Dag},
  title={Higher-order Computability},
  year={2015},
  publisher={Springer},
  series={Theory and Applications of Computability},
}

\bib{montahue}{article}{
  author={Montalb{\'a}n, Antonio},
  title={Open questions in reverse mathematics},
  journal={Bull. Symbolic Logic},
  volume={17},
  date={2011},
  number={3},
  pages={431--454},
}

\bib{mummy}{article}{
  author={Mummert, Carl},
  author={Simpson, Stephen G.},
  title={Reverse mathematics and $\Pi _2^1$ comprehension},
  journal={Bull. Symbolic Logic},
  volume={11},
  date={2005},
  number={4},
  pages={526--533},
}

\bib{wownelly}{article}{
  author={Nelson, Edward},
  title={Internal set theory: a new approach to nonstandard analysis},
  journal={Bull. Amer. Math. Soc.},
  volume={83},
  date={1977},
  number={6},
  pages={1165--1198},
}

\bib{robinson1}{book}{
  author={Robinson, Abraham},
  title={Non-standard analysis},
  publisher={North-Holland},
  place={Amsterdam},
  date={1966},
  pages={xi+293},
}

\bib{yamayamaharehare}{article}{
  author={Sakamoto, Nobuyuki},
  author={Yamazaki, Takeshi},
  title={Uniform versions of some axioms of second order arithmetic},
  journal={MLQ Math. Log. Q.},
  volume={50},
  date={2004},
  number={6},
  pages={587--593},
}

\bib{sayo}{article}{
  author={Sanders, Sam},
  author={Yokoyama, Keita},
  title={The {D}irac delta function in two settings of {R}everse {M}athematics},
  year={2012},
  journal={Archive for Mathematical Logic},
  volume={51},
  number={1},
  pages={99-121},
}

\bib{samGH}{article}{
  author={Sanders, Sam},
  title={The Gandy-Hyland functional and a hitherto unknown computational aspect of Nonstandard Analysis},
  year={2015},
  journal={Submitted, Available from: \url {http://arxiv.org/abs/1502.03622}},
}

\bib{samzoo}{article}{
  author={Sanders, Sam},
  title={The taming of the Reverse Mathematics zoo},
  year={2015},
  journal={Submitted, \url {http://arxiv.org/abs/1412.2022}},
}

\bib{sambon}{article}{
  author={Sanders, Sam},
  title={The unreasonable effectiveness of Nonstandard Analysis},
  year={2015},
  journal={Submitted, Available from: \url {http://arxiv.org/abs/1508.07434}},
}

\bib{samzooII}{article}{
  author={Sanders, Sam},
  title={The refining of the taming of the Reverse Mathematics zoo},
  year={2016},
  journal={To appear in Notre Dame Journal for Formal Logic, \url {http://arxiv.org/abs/1602.02270}},
}

\bib{simpson1}{collection}{
  title={Reverse mathematics 2001},
  series={Lecture Notes in Logic},
  volume={21},
  editor={Simpson, Stephen G.},
  publisher={ASL},
  place={La Jolla, CA},
  date={2005},
  pages={x+401},
}

\bib{simpson2}{book}{
  author={Simpson, Stephen G.},
  title={Subsystems of second order arithmetic},
  series={Perspectives in Logic},
  edition={2},
  publisher={CUP},
  date={2009},
  pages={xvi+444},
}

\bib{pimpson}{article}{
  author={Simpson, Stephen G.},
  author={Yokoyama, Keita},
  title={A nonstandard counterpart of \textsf {\textup {WWKL}}},
  journal={Notre Dame J. Form. Log.},
  volume={52},
  date={2011},
  number={3},
  pages={229--243},
}

\bib{troelstra1}{book}{
  author={Troelstra, Anne Sjerp},
  title={Metamathematical investigation of intuitionistic arithmetic and analysis},
  note={Lecture Notes in Mathematics, Vol.\ 344},
  publisher={Springer Berlin},
  date={1973},
  pages={xv+485},
}

\bib{kei1}{article}{
  author={Simpson, Stephen G.},
  author={Yokoyama, Keita},
  title={A nonstandard counterpart of WWKL},
  journal={Notre Dame J. Form. Log.},
  volume={52},
  date={2011},
  number={3},
  pages={229--243},
}

\bib{yuppie}{article}{
  author={Yu, Xiaokang},
  title={Lebesgue convergence theorems and reverse mathematics},
  journal={Math. Logic Quart.},
  volume={40},
  date={1994},
  number={1},
  pages={1--13},
}

\bib{yussie}{article}{
  author={Yu, Xiaokang},
  author={Simpson, Stephen G.},
  title={Measure theory and weak K\"onig's lemma},
  journal={Arch. Math. Logic},
  volume={30},
  date={1990},
  number={3},
  pages={171--180},
}


\end{biblist}
\end{bibdiv}

\bye



